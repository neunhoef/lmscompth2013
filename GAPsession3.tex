\documentclass[12pt]{article}

\pagestyle{empty}
\parindent0pt
%\usepackage{charter}
\usepackage{times}
\usepackage[a4paper,margin=1in]{geometry}
\newcommand{\GAP}{\textsf{GAP}}
\newcommand{\Stab}{\mbox{Stab}}
\newcommand{\hint}[1]{\par\textbf{Hint #1:}}
\newcommand{\see}[1]{$\to$ \texttt{?#1}}

\begin{document}
\begin{center}
\large LMS Short Course on Computational Group Theory

\Large Lab session 3

\large Using group libraries and conducting searches
\end{center}

\begin{enumerate}
\setlength{\parskip}{0pt}
\item How many non-abelian groups of order $24$ are there? (Of course, with
this formulation we always mean ``up to isomorphism''.)
\hint{} You want to use the \textbf{small groups library},
\see{NumberSmallGroups}, \see{AllSmallGroups}, \see{IsAbelian}
\item How many non-abelian groups of order $128$ are there?

Compute the average of the sizes of their centres.
\hint{1} To automate the counting you can use \see{List} and
\see{Collected} or use \see{Filtered} to get them all
\hint{2} For the centres the \see{arrow notation} could be convenient
\item What ID does the group generated by the three permutations
\[ (2,4,6,8,10), (1,9)(2,8)(3,7)(4,6)\ \mbox{and}\ 
(1,6)(2,7)(3,8)(4,9)(5,10) \]
have?
\hint{} \see{IdGroup}, \see{SmallGroup}
\item Find all composition series of all non-solvable groups of order $120$.
\hint{1} You can avoid making all small groups of order $120$ and 
then \texttt{Filtered} by using the more sophisticated
syntax of \see{AllSmallGroups}
\hint{2} Simply compute all normal subgroups of them using
\see{NormalSubgroups}
\item How many elements of order $3$ do all groups of order $48$ have
together? (Of course, we mean to take one group of each isomorphism
type.)
\hint{1} Fetch them all using \texttt{AllSmallGroups}, for each of them,
ask for all elements (\see{Elements}) and let \GAP\ count.
\hint{2} If you have a list \texttt{L} of groups, you can use a \see{for} loop
to run through all of them like this:

\texttt{for g in L do}

\hspace*{1cm} do stuff with g

\texttt{od;}
\hint{3} The same technique can be used to run through a list of elements.
Use \see{if} to decide, whether or not an element has order $3$ as in

\texttt{if Order(x) = 3 then}

\hspace*{1cm} increase a counter

\texttt{fi;}
\item Show that the Sylow $2$-subgroups of the Mathieu group $\mathrm{M}_{24}$ 
and the
sporadic simple Held group $\mathrm{He}$ are isomorphic.
\hint{1} \see{MathieuGroup}
\hint{2} Use the \textsf{AtlasRep} package and \see{AtlasGroup} to fetch
generators of $\mathrm{He}$ from the internet.
\hint{3} Use \see{SmallerDegreePermutationRepresentation} for the
Sylow $2$-subgroup of $\mathrm{He}$
\hint{4} Use \see{IsomorphismGroups}
\item Find the group with the fewest elements that is non-abelian, has 
trivial center and contains an element $a$ of order $2$ and an element
$b$ of order $3$ such that $ab$ has order $5$.
\end{enumerate}
\end{document}
