%\documentclass[handout]{beamer}
\documentclass{beamer}

\mode<presentation>
{
  %\usetheme{Warsaw}
  %\usetheme{Antibes}
  %\usetheme{Berkeley}
  %\usetheme{Copenhagen}
  %\usetheme{Hannover}
  %\usetheme{JuanLesPins}
  %\usetheme[left]{Marburg}
  %\usetheme{PaloAlto}
  %\usetheme{sidebar}
  \usetheme{CambridgeUS}
  %\useinnertheme[shadow=true]{rounded}
  %\usetheme{Singapore}
  %\usecolortheme{crane}
  %\usecolortheme{lily}
  %\usecolortheme{beetle}
  \usecolortheme{orchid}
  % oder ...
  %\setbeamercovered{transparent}
  % oder auch nicht
  %\setbeamercovered{transparent}
  % or whatever (possibly just delete it)
  \setbeamertemplate{navigation symbols}{}
  %\setbeamertemplate{blocks}[rounded][shadow=true]
  %\setbeamercolor{block body}{bg=bg=normal text.bg!70!black}
  %\setbeamercolor{subsection in sidebar}{fg=white}
}


\usepackage[british]{babel}
% oder was auch immer
\usepackage{amsmath}
\usepackage{amssymb}
\usepackage[noend]{algorithmic}
%\usepackage{algorithm}
\usepackage{stmaryrd}

\theoremstyle{definition}
%\newtheorem{lemma}[theorem]{Lemma}
\newtheorem{defprop}[theorem]{Definition/Proposition}
\newtheorem{question}[theorem]{Question}
\newtheorem{assumption}[theorem]{Assumption}
\newtheorem{construction}[theorem]{Construction}
\newtheorem{principle}[theorem]{Principle}
\newtheorem{idea}[theorem]{Idea}
\newtheorem{proposition}[theorem]{Proposition}
\newtheorem{algorithm}[theorem]{Algorithm}
%\newtheorem{problem}[theorem]{Problem}

\usepackage[latin1]{inputenc}
% oder was auch immer

\usepackage{times}
\usepackage[T1]{fontenc}
% Oder was auch immer. Zu beachten ist, das Font und Encoding passen
% m�ssen. Falls T1 nicht funktioniert, kann man versuchen, die Zeile
% mit fontenc zu l�schen.

%\usepackage[mtbold,subscriptcorrection,mtpluscal]{mathtime}

%\usepackage{graphicx}
%\usepackage{rotating}

\usepackage[all]{xy}

\newcommand{\Struc}[1]{{\color{structure}#1}}
\newcommand{\Alert}[1]{{\color{alert}#1}}

\newcommand{\F}{\mathbb{F}}
\newcommand{\N}{\mathbb{N}}
\newcommand{\Z}{\mathbb{Z}}
\newcommand{\R}{\mathbb{R}}
\newcommand{\Oh}{\mathcal{O}}
\newcommand{\Aut}{\mathsf{Aut}}
\newcommand{\Stab}{\mathsf{Stab}}
\newcommand{\ob}{\mathsf{Ob}}
\newcommand{\mor}{\mathsf{Mor}}
\newcommand{\PSL}{\mathsf{PSL}}
\newcommand{\cR}{\mathcal{R}}
\newcommand\cyclic\circlearrowleft

\newcommand{\mybar}[1]{\overline{\raisebox{1.2ex}{}#1}}
\newcommand{\mybaremp}{\mybar{\ \ }}

\usepackage{calc}

\newsavebox{\linksrausbox}
\newlength{\linksrauslen}
\newcommand{\linksraus}[1]{\sbox{\linksrausbox}{#1}%
\settowidth{\linksrauslen}{\usebox{\linksrausbox}\ }%
\usebox{\linksrausbox} \begin{minipage}[t]{\textwidth-\linksrauslen}}
\newcommand{\linksrausend}{\end{minipage}\par}%\smallskip}

\usepackage{pgf,pgfarrows,pgfnodes}

%\pgfdeclareimage[width=1mm]{checkmark}{checkmark}
%\newcommand{\eofr}[1]{\vfill\vspace*{-#1mm}\hfill\pgfuseimage{checkmark}
%\par\vspace*{#1mm}\vspace*{-4mm}}

\newcommand{\GAP}{\textsf{GAP}}

\pgfdeclareimage[width=0.5in]{univstandlogo}{univstandlogo}
%\title[Computing the 2-modular characters of Fi$_{23}$] 
%          % (optional, nur bei langen Titeln n�tig)
%          {Computing the 2-modular characters of Fi$_{23}$}
\title[Finitely presented groups 4]
{Finitely presented groups 4}

\author% (optional, nur bei vielen Autoren)
{Max Neunh�ffer}

\institute[University of St Andrews] % (optional, aber oft n�tig)
{ 
\pgfuseimage{univstandlogo} \\[5mm]
%University of St Andrews  \\[2mm]
}

\date[2 August 2013] % (optional, sollte der abgek�rzte Konferenzname sein)
{LMS Short Course on Computational Group Theory \\ 29 July -- 2 August 2013}

\begin{document}
%\newcmykcolor{MyRedViolet}{0.07 0.90 0 0.34}
\begin{frame}
  \titlepage
\end{frame}

\section{Small cancellation theory}

\subsection{The theory}

\begin{frame}
Let $G := \left< X \mid R \right>$ with $\hat X := X \cup X^{-1}$.
Assume that $R$ is \Struc{closed under rotation and inversion}
and all $r \in R$ are reduced.

\pause
\begin{definition}[Piece]
A \Struc{piece} (w.r.t.~$R$) is a nonempty word $p$ that is a prefix of
\Alert{two different} relators, i.e.: $pa,pb\in R$ for $a,b \in \hat
X^*$ with $a \neq b$.
\end{definition}

\pause
\begin{definition}[Condition $C'(\lambda)$]
We say $\left< X \mid R \right>$ is $C'(\lambda)$, if:
\begin{itemize}
\item for all $r=pa \in R$
where $p$ is a piece, we have $|p| < \lambda \cdot |r|$.
\end{itemize}
($|r|$ is the length in letters).
\end{definition}

\pause
\begin{definition}[Condition $T(q)$]
We say $\left<X \mid R \right>$ is $T(q)$, if the following holds:
\begin{itemize}
\item Let $3 \le h < q$ and $(r_1, r_2,
\ldots, r_h) \in R^h$ with \Struc{no successive elements} $r_i$, $r_{i+1}$ 
or $r_h, r_1$ an \Struc{inverse pair}. Then \Alert{at least one} of the products
$r_1r_2, r_2r_3, \ldots, r_hr_1$ is \Alert{reduced without cancellation}.
\end{itemize}
\end{definition}

\end{frame}

\begin{frame}
\begin{theorem}[Lyndon, Schupp]
Let $G = \left<X \mid R \right>$ with $R$ \Struc{closed under rotation
and inversion} and all $r \in R$ are reduced. If $\left< X \mid R
\right>$ fulfills \Alert{at least one of}:
\begin{itemize}
\item $C'(1/6)$ and $T(3)$, or
\item $C'(1/4)$ and $T(4)$, or
\item $C'(1/3)$ and $T(6)$,
\end{itemize}
then \Struc{Dehn's algorithm solves the word problem for $G$}.
\end{theorem}

\pause
What is Dehn's algorithm?

\pause
What does this mean for the structure of $G$?

\pause
\bigskip
\begin{definition}[Dehn RWS]
Write all $r \in R$ as $r=ab$ with $|a|>|b|$ and define a rule $a \to
b^{-1}$.
\end{definition}

\end{frame}

\subsection{Dehn's algorithm}

\begin{frame}
\begin{algorithm}[Dehn's algorithm]
Let $G = \left<X \mid R\right>$ and let $\cR$ be a length-reducing RWS 
for $\hat X = X \cup X^{-1}$.

\begin{enumerate}
\item \textbf{Input:} a \Struc{word $w \in \hat X^*$}.
\item \Struc{Freely reduce} $w$.
\item If any rewrite rule matches, \Struc{apply} it and \Struc{go back} to 2.
\item \textbf{Output:} the new $w$.
\end{enumerate}
\pause
\textbf{Note} that 3.~is \Alert{not deterministic}.
\end{algorithm}

\pause
Saying that \Struc{``Dehn's algorithm solves the word problem''}
means:
\begin{itemize}
\item The output is the empty word $\varepsilon$ if and only if $w=_G
1$,
\item \Alert{not depending on which rewrite is applied in 3.}
\end{itemize}

\pause
\textbf{Note:} 
\begin{itemize}
\item For a general RWS, this \Alert{does not make sense at all}.
\pause
\item If $w \neq_G 1$, then the output \Alert{can be different},
depending on the choice in 3.
\pause
\item For a word of length $n$, this terminates in \Alert{at most $n$
steps}.
\end{itemize}
\end{frame}

\section{Hyperbolic groups}

\subsection{Linear Dehn functions}

\begin{frame}
If $ab \in R$ with $|a|>|b|$ and $w=xay$, then Dehn rewrites
this to $xb^{-1}y$.

\pause
Thus: $w=x(ab)x^{-1}\ xb^{-1}y$

\pause
So $w$ is written as a conjugate of a relator times a shorter word.

\begin{lemma}
If $G = \left< X \mid R \right>$ is small cancellation, then Dehn
works and every word $w \in \hat X^*$ of length $n$ that
is equal to $1$ in $G$ is the product of at most $n$ conjugates of a
relator. Thus, the Dehn function $\delta(n) \le n \ \forall n$.
\end{lemma}

\pause
\begin{definition}[Hyperbolic group]
A group is called \Struc{hyperbolic}, if it has a finite presentation
with a Dehn function that is \Struc{bounded by a linear function}.
\end{definition}

\pause
\textbf{We have} for a group $G = \left< X \mid R \right>$:
\begin{eqnarray*}
 \mbox{small cancellation} &\Longrightarrow& \mbox{Dehn's algorithm works}
   \Longrightarrow \mbox{hyperbolic} \\
   \pause &\Longrightarrow&
   \mbox{has presentation with a working Dehn} 
\end{eqnarray*}

\end{frame}

\section{Finite state automata}

\subsection{Definition by picture}

\begin{frame}
FSA, definition by picture, regular languages --> regular
    expressions

\end{frame}

\subsection{Using a FSA for a RWS}

\begin{frame}
Explain FSA for a RWS
\end{frame}

\subsection{Cyclic words and conjugacy classes}

\begin{frame}
Cyclic words -> Conjugacy classes
\end{frame}

\section{Automatic groups}

\subsection{Definition}

\begin{frame}
Word acceptors
\end{frame}

\begin{frame}
Def automatic group: mupltiplier automata
\end{frame}

\begin{frame}
The class of automatic groups
\end{frame}

\section{What we left out}

\begin{frame}
Parallelisation

Quotient algorithms: Nilpotent, Soluble, $p$-Quotient

Polycyclic groups

Finding presentations from other representations

Symmetric presentations

infinite presentations

laws

new developments in algorithmic small cancellation theory
\end{frame}

\end{document}

