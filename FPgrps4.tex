%\documentclass[handout]{beamer}
\documentclass{beamer}

\mode<presentation>
{
  %\usetheme{Warsaw}
  %\usetheme{Antibes}
  %\usetheme{Berkeley}
  %\usetheme{Copenhagen}
  %\usetheme{Hannover}
  %\usetheme{JuanLesPins}
  %\usetheme[left]{Marburg}
  %\usetheme{PaloAlto}
  %\usetheme{sidebar}
  \usetheme{CambridgeUS}
  %\useinnertheme[shadow=true]{rounded}
  %\usetheme{Singapore}
  %\usecolortheme{crane}
  %\usecolortheme{lily}
  %\usecolortheme{beetle}
  \usecolortheme{orchid}
  % oder ...
  %\setbeamercovered{transparent}
  % oder auch nicht
  %\setbeamercovered{transparent}
  % or whatever (possibly just delete it)
  \setbeamertemplate{navigation symbols}{}
  %\setbeamertemplate{blocks}[rounded][shadow=true]
  %\setbeamercolor{block body}{bg=bg=normal text.bg!70!black}
  %\setbeamercolor{subsection in sidebar}{fg=white}
}


\usepackage[british]{babel}
% oder was auch immer
\usepackage{amsmath}
\usepackage{amssymb}
\usepackage[noend]{algorithmic}
%\usepackage{algorithm}
\usepackage{stmaryrd}

\theoremstyle{definition}
%\newtheorem{lemma}[theorem]{Lemma}
\newtheorem{defprop}[theorem]{Definition/Proposition}
\newtheorem{question}[theorem]{Question}
\newtheorem{assumption}[theorem]{Assumption}
\newtheorem{construction}[theorem]{Construction}
\newtheorem{principle}[theorem]{Principle}
\newtheorem{idea}[theorem]{Idea}
\newtheorem{proposition}[theorem]{Proposition}
\newtheorem{algorithm}[theorem]{Algorithm}
%\newtheorem{problem}[theorem]{Problem}

\usepackage[latin1]{inputenc}
% oder was auch immer

\usepackage{times}
\usepackage[T1]{fontenc}
% Oder was auch immer. Zu beachten ist, das Font und Encoding passen
% m�ssen. Falls T1 nicht funktioniert, kann man versuchen, die Zeile
% mit fontenc zu l�schen.

%\usepackage[mtbold,subscriptcorrection,mtpluscal]{mathtime}

%\usepackage{graphicx}
%\usepackage{rotating}

\usepackage[all]{xy}

\newcommand{\Struc}[1]{{\color{structure}#1}}
\newcommand{\Alert}[1]{{\color{alert}#1}}

\newcommand{\F}{\mathbb{F}}
\newcommand{\N}{\mathbb{N}}
\newcommand{\Z}{\mathbb{Z}}
\newcommand{\R}{\mathbb{R}}
\newcommand{\Oh}{\mathcal{O}}
\newcommand{\Aut}{\mathsf{Aut}}
\newcommand{\Stab}{\mathsf{Stab}}
\newcommand{\ob}{\mathsf{Ob}}
\newcommand{\mor}{\mathsf{Mor}}
\newcommand{\PSL}{\mathsf{PSL}}
\newcommand{\cR}{\mathcal{R}}
\newcommand\cyclic\circlearrowleft
\newcommand{\tvs}{\textvisiblespace}

\newcommand{\mybar}[1]{\overline{\raisebox{1.2ex}{}#1}}
\newcommand{\mybaremp}{\mybar{\ \ }}

\usepackage{calc}

\newsavebox{\linksrausbox}
\newlength{\linksrauslen}
\newcommand{\linksraus}[1]{\sbox{\linksrausbox}{#1}%
\settowidth{\linksrauslen}{\usebox{\linksrausbox}\ }%
\usebox{\linksrausbox} \begin{minipage}[t]{\textwidth-\linksrauslen}}
\newcommand{\linksrausend}{\end{minipage}\par}%\smallskip}

\usepackage{pgf,pgfarrows,pgfnodes}

%\pgfdeclareimage[width=1mm]{checkmark}{checkmark}
%\newcommand{\eofr}[1]{\vfill\vspace*{-#1mm}\hfill\pgfuseimage{checkmark}
%\par\vspace*{#1mm}\vspace*{-4mm}}

\newcommand{\GAP}{\textsf{GAP}}

\pgfdeclareimage[width=0.5in]{univstandlogo}{univstandlogo}
%\title[Computing the 2-modular characters of Fi$_{23}$] 
%          % (optional, nur bei langen Titeln n�tig)
%          {Computing the 2-modular characters of Fi$_{23}$}
\title[Finitely presented groups 4]
{Finitely presented groups 4}

\author% (optional, nur bei vielen Autoren)
{Max Neunh�ffer}

\institute[University of St Andrews] % (optional, aber oft n�tig)
{ 
\pgfuseimage{univstandlogo} \\[5mm]
%University of St Andrews  \\[2mm]
}

\date[2 August 2013] % (optional, sollte der abgek�rzte Konferenzname sein)
{LMS Short Course on Computational Group Theory \\ 29 July -- 2 August 2013}

\begin{document}
%\newcmykcolor{MyRedViolet}{0.07 0.90 0 0.34}
\begin{frame}
  \titlepage
\end{frame}

\section{Small cancellation theory}

\subsection{The theory}

\begin{frame}
Let $G := \left< X \mid R \right>$ with $\hat X := X \cup X^{-1}$.
Assume that $R$ is \Struc{closed under rotation and inversion}
and all $r \in R$ are reduced.

\pause
\begin{definition}[Piece]
A \Struc{piece} (w.r.t.~$R$) is a nonempty word $p$ that is a prefix of
\Alert{two different} relators, i.e.: $pa,pb\in R$ for $a,b \in \hat
X^*$ with $a \neq b$.
\end{definition}

\pause
\begin{definition}[Condition $C'(\lambda)$]
We say $\left< X \mid R \right>$ is $C'(\lambda)$, if:
\begin{itemize}
\item for all $r=pa \in R$
where $p$ is a piece, we have $|p| < \lambda \cdot |r|$.
\end{itemize}
($|r|$ is the length in letters).
\end{definition}

\pause
\begin{definition}[Condition $T(q)$]
We say $\left<X \mid R \right>$ is $T(q)$, if the following holds:
\begin{itemize}
\item Let $3 \le h < q$ and $(r_1, r_2,
\ldots, r_h) \in R^h$ with \Struc{no successive elements} $r_i$, $r_{i+1}$ 
or $r_h, r_1$ an \Struc{inverse pair}. Then \Alert{at least one} of the products
$r_1r_2, r_2r_3, \ldots, r_hr_1$ is \Alert{reduced without cancellation}.
\end{itemize}
\end{definition}

\end{frame}

\begin{frame}
\begin{theorem}[Lyndon, Schupp]
Let $G = \left<X \mid R \right>$ with $R$ \Struc{closed under rotation
and inversion} and all $r \in R$ are reduced. If $\left< X \mid R
\right>$ fulfills \Alert{at least one of}:
\begin{itemize}
\item $C'(1/6)$ and $T(3)$, or
\item $C'(1/4)$ and $T(4)$, or
\item $C'(1/3)$ and $T(6)$,
\end{itemize}
then \Struc{Dehn's algorithm solves the word problem for $G$}.
\end{theorem}

\pause
What is Dehn's algorithm?

\pause
What does this mean for the structure of $G$?

\pause
\bigskip
\begin{definition}[Dehn RWS]
Write all $r \in R$ as $r=ab$ with $|a|>|b|$ and define a rule $a \to
b^{-1}$.
\end{definition}

\end{frame}

\subsection{Dehn's algorithm}

\begin{frame}
\begin{algorithm}[Dehn's algorithm]
Let $G = \left<X \mid R\right>$ and let $\cR$ be a length-reducing RWS 
for $\hat X = X \cup X^{-1}$.

\begin{enumerate}
\item \textbf{Input:} a \Struc{word $w \in \hat X^*$}.
\item \Struc{Freely reduce} $w$.
\item If any rewrite rule matches, \Struc{apply} it and \Struc{go back} to 2.
\item \textbf{Output:} the new $w$.
\end{enumerate}
\pause
\textbf{Note} that 3.~is \Alert{not deterministic}.
\end{algorithm}

\pause
Saying that \Struc{``Dehn's algorithm solves the word problem''}
means:
\begin{itemize}
\item The output is the empty word $\varepsilon$ if and only if $w=_G
1$,
\item \Alert{not depending on which rewrite is applied in 3.}
\end{itemize}

\pause
\textbf{Note:} 
\begin{itemize}
\item For a general RWS, this \Alert{does not make sense at all}.
\pause
\item If $w \neq_G 1$, then the output \Alert{can be different},
depending on the choice in 3.
\pause
\item For a word of length $n$, this terminates after \Alert{at most $n$
steps}.
\end{itemize}
\end{frame}

\section{Hyperbolic groups}

\subsection{Linear Dehn functions}

\begin{frame}
If $ab \in R$ with $|a|>|b|$ and $w=xay$, then Dehn rewrites
this to $xb^{-1}y$.

\pause
Thus: $w=x(ab)x^{-1}\ xb^{-1}y$

\pause
So $w$ is written as a \Struc{conjugate of a relator} times a shorter word.

\begin{lemma}
If $G = \left< X \mid R \right>$ is small cancellation, then Dehn
works and every word $w \in \hat X^*$ of length $n$ that
is equal to $1$ in $G$ is the product of at most $n$ conjugates of a
relator. Thus, the \Struc{Dehn function} $\delta(n) \le n$ for all $n$.
\end{lemma}

\pause
\begin{definition}[Hyperbolic group]
A group is called \Struc{hyperbolic}, if it has a finite presentation
with a Dehn function that is \Struc{bounded by a linear function}.
\end{definition}

\pause
\textbf{We have} for a group $G = \left< X \mid R \right>$:

\vspace*{-3mm}
\begin{eqnarray*}
 \mbox{small cancellation} &\Longrightarrow& \mbox{Dehn's algorithm works}
   \Longrightarrow \mbox{hyperbolic} \\
   \pause &\Longrightarrow&
   \mbox{has presentation with a working Dehn} 
\end{eqnarray*}

\pause
\textbf{Question:} How do we \Struc{execute Dehn's algorithm}
\Alert{efficiently}?
\end{frame}

\section{Finite state automata}

\subsection{Definition by picture}

\pgfdeclareimage[width=4in]{fsa}{fsa}
\pgfdeclareimage[width=4in]{nondetfsa}{nondetfsa}
\begin{frame}
\begin{center}
\pgfuseimage{fsa}
\end{center}

\pause
Every path in this digraph has a \Struc{label}.

\pause
There is one \Struc{start state} and some \Struc{accept states}.
\pause
\[ \mathcal{L} := \left\{ \mbox{labels of paths from start to an accept
state} \right\} \subseteq \{ X,Y,Z \}^* \]

\pause
This is a \Alert{regular language}: $XY^*Z(Y+X) + YZ(YZ)^*(Z+XZ^*Y)$.
\end{frame}

\begin{frame}
\begin{center}
\pgfuseimage{nondetfsa}
\end{center}
\textbf{Non-deterministic variants:}
\begin{itemize}
\item Allow \Struc{empty} (or $\varepsilon$) transitions.
\pause
\item Allow \Struc{more than one} transition \Alert{with the same label} 
leaving a state.
\end{itemize}
\pause
\Alert{However:} \begin{minipage}[t]{3.5in}
The classes of languages of deterministic and
nondeterministic finite state automata are the same.
\end{minipage}

\end{frame}

\subsection{Using a FSA for a RWS}

\begin{frame}
Assume $\cR$ is a RWS and assume for simplicity 
that \Struc{no left hand side (LHS) of a rewrite is properly contained in
another one}.

\pause
\begin{definition}[FSA for a RWS]
\textbf{States:}

Define a \Struc{state} for every \Alert{prefix of a LHS of a
rewrite}.

\pause
The \Struc{empty prefix} is the \Alert{start state}.

\pause
The \Struc{complete LHSs} are the \Alert{accept states}.

\smallskip
\pause
\textbf{Transitions:}

If XY is a \Struc{non-accepting state}, then there is a transition 
labelled with ``Z'' to XYZ \Alert{if this is still a prefix} of a LHS.

\pause
If XYZ is \Struc{not a prefix}, then there is a transition 
labelled with ``Z'' to the \Alert{longest suffix of XYZ that is a prefix}
of a LHS.

\pause
\smallskip
This defines a \Struc{deterministic FSA} which recognises LHSs.
\end{definition}

\pause
$\Longrightarrow$ \Struc{Very fast algorithm} to \Alert{recognise
rewrite rules that apply}.

\pause
$\Longrightarrow$ \Struc{Crucial step for} \Alert{Dehn's algorithm}.
\end{frame}

\section{Automatic groups}

\subsection{Definition}

\begin{frame}
Sometimes, we want to describe \Struc{relations} on $\hat X^*$ by a FSA:

\pause
\begin{definition}[2-variable FSA by padding]
Define $p : \hat X^* \times \hat X^* \to 
((\hat X \cup \{\$\}) \times (\hat X \cup \{\$\}))^*$ by \Struc{padding
the shorter word} at the end with $\$$ symbols:

\vspace*{-8mm}
\begin{eqnarray*}
p( ABC, DEFGH ) &=& (A,D)(B,E)(C,F)(\$,G)(\$,H) \\
   p( ABC, D ) &=& (A,D)(B,\$)(C,\$)
\end{eqnarray*}

\vspace*{-3mm}
\pause
A FSA with \Struc{alphabet} $\hat X \cup \{\$\}$ accepts a pair
$(v,w) \in \hat X^* \times \hat X^*$ iff there is a path from the start
state to an accept state with label $p(v,w)$.
\end{definition}

\pause
We prepare ourselves for the definition of \Struc{automatic groups}:

\pause
\begin{definition}[Word acceptor]
Let $G = \left< X \mid R \right>$ and $\hat X := X \cup X^{-1}$. A FSA on
$\hat X$ is called a \Struc{word acceptor} for $G$, if it
accepts \Alert{at least one} word for each element of $G$.
\pause

It is called a \Struc{unique word acceptor}, if it accepts
\Alert{exactly one} word for each element of $G$.
\end{definition}
\end{frame}

\begin{frame}
\begin{definition}[Automatic group]
Let $G$ be a group that is generated as a monoid by the set $\hat X$.
Then $G$ is \Struc{automatic w.r.t.~$\hat X$}, if there exist FSA $W$
and $M_x$ for $x \in \hat X \cup \{\varepsilon\}$, s.th.:
\begin{itemize}
\item $W$ has alphabet $\hat X$ and is a \Struc{word acceptor} for $G$, and
\pause
\item $M_x$ has alphabet $\hat X \cup \{\$\}$ and $(v,w) \in \hat X^*
\times \hat X^*$ (where $v$ and $w$ are accepted by $W$) 
is \Alert{accepted} by $M_x$ iff \Alert{$vx =_G w$}.
\end{itemize}
\pause
The automata $W$ and $M_x$ are called an \Struc{automatic structure} for
$G$, the $M_x$ are the \Struc{multiplier automata}.
\end{definition}

\pause
\begin{theorem}[Epstein et al.~1992]
Being automatic is a \Struc{property of $G$} and not of $\hat X$.
\end{theorem}

\pause
\begin{definition}[Shortlex automatic structure]
If $W$ accepts
\Alert{precisely the shortlex minimal words of $\hat X^*$} 
for the elements of $G$, then
$(W,\{M_x\})$ is a \Struc{shortlex automatic structure}.
\end{definition}
\end{frame}

\subsection{Computing automatic structures}

\begin{frame}
\begin{definition}[Word differences]
Let $v,w \in \hat X^*$ and let $v_i$ be the \Struc{prefix of $v$ of length
$i$}. The \Struc{word differences} of $v$ and $w$ are
$D(v,w) := \{ v_i^{-1} w_i \mid i \in \N \} \subseteq G$.

\pause
Note that all $D(v,w)$ are \Alert{finite sets}.
\end{definition}
\pause

\begin{theorem}
Let $(W,\{M_x\})$ be an \Struc{automatic structure}. The set

\vspace*{-4mm}
\[ D := \bigcup_{(v,w) \mbox{ \scriptsize accepted by some } M_x} D(v,w)\]

\vspace*{-4mm}
is \Alert{finite}.
\end{theorem}
\end{frame}

\begin{frame}
\begin{block}{Idea of shortlex automatic structure computation}
Let $G = \left< X \mid R \right>$ and set $\hat X := X \cup X^{-1}$.
\begin{enumerate}
\item Run a \Struc{shortlex Knuth-Bendix} on a RWS coming from the monoid
presentation.
\pause
\item \Struc{Stop} after some time, \Alert{even} if it has \Alert{not
completed}.
\pause
\item \Struc{Compute} word differences as above, and \Struc{approximate} 
FSA to recognise them.
\pause
\item \Struc{Compute} a \Alert{candidate} for the word acceptor $W$.
\pause
\item \Struc{Compute} \Alert{candidates} for the multiplier FSA $M_x$.
\pause
\item \Struc{Carry out} \Alert{correctness tests}, terminate if OK,
otherwise go back.
\end{enumerate}
\end{block}

\pause
\vspace*{5mm}
\centerline{\url{http://tinyurl.com/MNGAPsess/GAP\_FP\_9.g}}
% Show a successful Knuth-Bendix run with KBMAG
\end{frame}

\subsection{Properties}
\begin{frame}
The class of automatic groups is
\begin{itemize}
\item \Struc{closed} under taking \Alert{direct products},
\pause
\item \Struc{closed} under taking \Alert{free products with finite
amalgamated subgroup},
\pause 
\item \Struc{closed} under taking \Alert{HNN-extensions with finite
conjugated subgroup}.
\end{itemize}

\pause
Furthermore:
\begin{itemize}
\item \Struc{Hyperbolic groups} are \Alert{automatic}.
\pause
\item \Struc{Free factors} of automatic groups are automatic.
\pause
\item It has \Alert{not been proved} that \Struc{direct factors} of
automatic groups are automatic.
\pause
\item If $[G:H] < \infty$, then $G$ is automatic iff $H$ is.
\end{itemize}

\pause
\textbf{Thus:} 

\centerline{\Struc{Automatic groups} are a large class of
groups with \Alert{solvable word problem}.}
\end{frame}

\section{What we left out}

\begin{frame}
In this lecture series we have \Alert{not mentioned} a lot of topics:

\pause
\begin{itemize}
\item \Struc{Polycyclic groups} (see Bettina's series)
\pause
\item \Struc{Parallelisation} of algorithms
\pause
\item \Struc{Quotient algorithms}: Nilpotent, Soluble, $p$-Quotient
\pause
\item \Struc{Finding matrix representations} (see W.~Plesken et al.)
\pause
\item \Struc{Finding presentations} if group is given in 
\Alert{another representation}
\pause
\item \Struc{Symmetric presentations} (see R.~Curtis et al.)
\pause
\item \Struc{Infinite presentations}
\pause
\item \Struc{Laws} (e.g.~Burnside groups and algorithms for such problems)
\pause
\item \Struc{New developments} in \Alert{algorithmic small cancellation
theory} 

(see Richard's talk yesterday)
\end{itemize}

\bigskip
\pause
Derek F.~Holt, Bettina Eick, Eamonn A.~O'Brien: 

\hspace*{1cm}``Handbook of Computational Group Theory''
\end{frame}

\end{document}

