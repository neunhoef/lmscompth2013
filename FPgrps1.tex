%\documentclass[handout]{beamer}
\documentclass{beamer}

\mode<presentation>
{
  %\usetheme{Warsaw}
  %\usetheme{Antibes}
  %\usetheme{Berkeley}
  %\usetheme{Copenhagen}
  %\usetheme{Hannover}
  %\usetheme{JuanLesPins}
  %\usetheme[left]{Marburg}
  %\usetheme{PaloAlto}
  %\usetheme{sidebar}
  \usetheme{CambridgeUS}
  %\useinnertheme[shadow=true]{rounded}
  %\usetheme{Singapore}
  %\usecolortheme{crane}
  %\usecolortheme{lily}
  %\usecolortheme{beetle}
  \usecolortheme{orchid}
  % oder ...
  %\setbeamercovered{transparent}
  % oder auch nicht
  %\setbeamercovered{transparent}
  % or whatever (possibly just delete it)
  \setbeamertemplate{navigation symbols}{}
  %\setbeamertemplate{blocks}[rounded][shadow=true]
  %\setbeamercolor{block body}{bg=bg=normal text.bg!70!black}
  %\setbeamercolor{subsection in sidebar}{fg=white}
}


\usepackage[british]{babel}
% oder was auch immer
\usepackage{amsmath}
\usepackage{amssymb}
\usepackage[noend]{algorithmic}
%\usepackage{algorithm}

\theoremstyle{definition}
%\newtheorem{lemma}[theorem]{Lemma}
\newtheorem{defprop}[theorem]{Definition/Proposition}
\newtheorem{question}[theorem]{Question}
\newtheorem{assumption}[theorem]{Assumption}
\newtheorem{construction}[theorem]{Construction}
\newtheorem{principle}[theorem]{Principle}
\newtheorem{idea}[theorem]{Idea}
%\newtheorem{problem}[theorem]{Problem}
\usepackage[latin1]{inputenc}
% oder was auch immer

\usepackage{times}
\usepackage[T1]{fontenc}
% Oder was auch immer. Zu beachten ist, das Font und Encoding passen
% m�ssen. Falls T1 nicht funktioniert, kann man versuchen, die Zeile
% mit fontenc zu l�schen.

%\usepackage[mtbold,subscriptcorrection,mtpluscal]{mathtime}

%\usepackage{graphicx}
%\usepackage{rotating}

\newcommand{\Struc}[1]{{\color{structure}#1}}
\newcommand{\Alert}[1]{{\color{alert}#1}}

\newcommand{\F}{\mathbb{F}}
\newcommand{\N}{\mathbb{N}}
\newcommand{\Z}{\mathbb{Z}}
\newcommand{\R}{\mathbb{R}}
\newcommand{\Oh}{\mathcal{O}}
\newcommand{\Aut}{\mathsf{Aut}}
\newcommand{\ob}{\mathsf{Ob}}
\newcommand{\mor}{\mathsf{Mor}}
\newcommand{\PSL}{\mathsf{PSL}}
\newcommand{\cR}{\mathcal{R}}
\newcommand\cyclic\circlearrowleft

\newcommand{\mybar}[1]{\overline{\raisebox{1.2ex}{}#1}}
\newcommand{\mybaremp}{\mybar{\ \ }}

\usepackage{calc}

\newsavebox{\linksrausbox}
\newlength{\linksrauslen}
\newcommand{\linksraus}[1]{\sbox{\linksrausbox}{#1}%
\settowidth{\linksrauslen}{\usebox{\linksrausbox}\ }%
\usebox{\linksrausbox} \begin{minipage}[t]{\textwidth-\linksrauslen}}
\newcommand{\linksrausend}{\end{minipage}\par}%\smallskip}

\usepackage{pgf,pgfarrows,pgfnodes}

%\pgfdeclareimage[width=1mm]{checkmark}{checkmark}
%\newcommand{\eofr}[1]{\vfill\vspace*{-#1mm}\hfill\pgfuseimage{checkmark}
%\par\vspace*{#1mm}\vspace*{-4mm}}

\newcommand{\GAP}{\textsf{GAP}}

\pgfdeclareimage[width=0.5in]{univstandlogo}{univstandlogo}
%\title[Computing the 2-modular characters of Fi$_{23}$] 
%          % (optional, nur bei langen Titeln n�tig)
%          {Computing the 2-modular characters of Fi$_{23}$}
\title[Finitely presented groups 1]
{Finitely presented groups 1}

\author% (optional, nur bei vielen Autoren)
{Max Neunh�ffer}

\institute[University of St Andrews] % (optional, aber oft n�tig)
{ 
\pgfuseimage{univstandlogo} \\[5mm]
%University of St Andrews  \\[2mm]
}

\date[29 July 2013] % (optional, sollte der abgek�rzte Konferenzname sein)
{LMS Short Course on Computational Group Theory \\ 29 July -- 2 August 2013}

\begin{document}
%\newcmykcolor{MyRedViolet}{0.07 0.90 0 0.34}
\begin{frame}
  \titlepage
\end{frame}

\section{Finitely presented groups}

\subsection{Introduction and definition}

\begin{frame}
$\left< C \mid C^k=1 \right>$ \pause is the \Struc{cyclic group} of
order $k$.

\pause
\bigskip
$\left< A, B \mid ABA^{-1}B^{-1} \right>$ \pause is the
\Struc{integral lattice $\Z^2$} with addition.

\pause
\begin{block}{``Definition''}
$G := \left< X \mid R_1, R_2, \ldots, R_k\right>$ is the
``\Alert{largest}'' 
group that has $X$ as \Struc{generating system}, such that
the \Struc{relations} $R_1, \ldots, R_k$ hold.

\pause
\smallskip
\Alert{Every} group with a generating system $X$, for which the relations $R_1,
\ldots, R_k$ hold, is a \Struc{quotient of $G$.}
\end{block}

\pause
\bigskip
$\left< P, Q \mid P^n, Q^2, PQPQ \right>$ \pause is the
\Struc{dihedral group} of order $2n$.

\pause
\bigskip
$\left< S, T \mid S^3, T^2 \right>$ is the \Struc{modular group
$\PSL(2,\Z)$}.

\pause
\bigskip
$\left< D, E, F \mid DEFED, EFD, D^2EF\right>$ is the \Struc{trivial
group $\{1\}$}.
\end{frame}
        
\begin{frame}
$X$: finite set of \Struc{generators}, $\hat X := X
\stackrel{.}{\cup} X^{-1}$ with
involution $x \mapsto x^{-1}$

\smallskip
\pause
$\hat X^*$: set of \Alert{finite words} in $\hat X$, 
concatenation as product

\smallskip
\pause
$R \subseteq \hat X^*$: finite set of \Struc{relations}

\medskip
\pause
\Struc{Define} $\sim$ as  \Alert{finest equivalence relation} on $\hat X^*$ 
with
\begin{itemize}
\item $uv \sim uxx^{-1}v$ for all $u,v \in \hat X^*$ and all $x \in
\hat X$, and
\item $uv \sim urv$ for all $u,v \in \hat X^*$ and all $r \in R$.
\end{itemize}

\pause
\begin{defprop}[Finitely presented group]
The set $G := \hat X^*/\sim$ of equivalence classes is a group with
concatenation as product. 
It is called \Struc{the finitely presented group with
generators $X$ and relations $R$} and is denoted by
\Alert{$G := \left< X \mid R\right>$}.
\pause
$\mybaremp:\hat X^* \to G, w \mapsto \mybar{w}$ is the natural map from $w$
to its equivalence class.
\end{defprop}

\bigskip
\pause e.g.~in $G=\left< S,T \mid S^6, T^2, STST\right>$:
\[ \emptyset \sim \Struc{STST}
   \pause    \sim \Struc{S}STST\Struc{TST}
   \pause    = SSTS\Alert{TT}ST
   \pause    \sim SSTSST  = S^2TS^2T\]
\end{frame}

\subsection{Universal property}

\begin{frame}
\begin{theorem}[Universal property (Dyck, 1882)] 
Let $G := \left< X \mid R \right>$ be a finitely presented group.

Let $H$ be any group, $f : X \to H$ \Struc{any map} and set $f(x^{-1}) :=
f(x)^{-1}$ for $x \in X$. If 
\[ f(x_1) \cdot f(x_2) \cdot \cdots \cdot f(x_k) = 1 \in H \]
for all $r = x_1 x_2 \cdots x_k \in R$,
then there is a \Struc{unique group homomorphism} $\tilde f : G \to H$
with $\tilde f(\mybar x) = f(x)$ for all $x \in X$.

\pause
In other words: \Struc{any map $f : X \to H$} can be \Alert{extended
to a group homomorphism $\tilde f: G \to H$ in a unique way}.
\end{theorem}

\pause
\Struc{This means:} $G$ has all groups as a \Struc{quotient} that
have a generating system $X$ which \Alert{fulfills the relations} $R$.

\medskip
\pause
$\left< A, B \mid \right>$ is the \Struc{free group on two
generators}, it has any 2-generated group as a quotient.

\medskip
\pause
$\left< C \mid C^k \right>$ has all cyclic groups of order $o$
dividing $k$ as quotients.
\end{frame}

\subsection{Fundamental problems}

\begin{frame}
In a seminal paper, Max Dehn formulated three \Struc{fundamental
problems}:

\begin{problem}[Word problem (Dehn 1911)]
For $G := \left< X \mid R \right>$ and a \Struc{word $w \in \hat X^*$},
\Alert{decide in finitely many steps} whether or not $\mybar w = 1 \in
G$.
\end{problem}

\pause
\begin{problem}[Conjugacy problem (Dehn 1911)]
For $G := \left< X \mid R \right>$ and two \Struc{words $v,w \in \hat X^*$},
\Alert{decide in finitely many steps} whether or not there is a word
$u \in \hat X^*$ such that $\mybar w = \mybar{u v u^{-1}} \in G$.
\end{problem}

\pause
\begin{problem}[Isomorphism problem (Dehn 1911)]
For $G := \left< X \mid R \right>$ and $H := \left< Y \mid S \right>$,
\Alert{decide in finitely many steps} whether or not $G \cong H$,
\pause \Struc{or even}: for a map $f : X \to \hat Y^*$, 
\Alert{decide in finitely
many steps} whether or not $f$ induces an isomorphism $\tilde f : G \to H$.
\end{problem}

\pause
\smallskip
We would like to have \Struc{algorithms} for these decision problems
$\ldots$

\end{frame}

\begin{frame}
\begin{problem}[Word problem (Dehn 1911)]
For $G := \left< X \mid R \right>$ and a \Struc{word $w \in \hat X^*$},
\Alert{decide in finitely many steps} whether or not $\mybar w = 1 \in
G$.
\end{problem}

\pause
\begin{lemma}
Let $G=\left< X \mid R\right>$, then $\mybar{w}=1$ for a $w \in \hat X^*$
if and only if

\vspace*{-3.5mm}
\[ w \sim \prod_{i=1}^k w_i^{-1} r_i w_i \qquad 
\Alert{\mbox{in } F=\left<X \mid \right>} \]

\vspace*{-1.5mm}
for a $k \in \N$ and some $w_i \in \hat X^*$ and $r_i \in R$. 

\pause
\smallskip
For $w$ with $\mybar w=1$ let $A(w)$ be the \Struc{minimal such $k$}.
\end{lemma}

\pause
\begin{definition}[Dehn function]
\[ \delta(\ell) := \max\{ A(w) \mid w \in \hat X^* \mbox{ of length } \ell 
\mbox{ and } \mybar w = 1\} \]
\end{definition}

\end{frame}

\subsection{The word problem is insolvable}

\begin{frame}
\begin{theorem}[Novikov 1955, Boone 1958]
There is a finite presentation
$G=\left< X \mid R \right>$, for which it is \Alert{proven}
that \Struc{there is no algorithm} to solve the word problem.
\end{theorem}

\bigskip
\pause
\begin{theorem}[Shapiro]
The word problem for
$G=\left< X \mid R \right>$ is \Struc{algorithmically solvable}
if and only if the Dehn function $\delta$ is \Alert{recursively computable}.
\end{theorem}

\bigskip
\pause
\begin{center}
The \Struc{growth rate of $\delta$} is a measure for the
\Alert{difficulty of the word problem}.
\end{center}

\bigskip
\pause
The \Struc{conjugacy problem} and the \Struc{isomorphism problem} are
\Alert{unsolvable} in the same way. \pause 

Even the question if $\left< X \mid R \right> = \{ 1 \}$ is
in general \Alert{unsolvable}.
\end{frame}

% GAP session here to type in some FP groups and ask some questions,
% show that something works OK, other things do not seem to work.

\section{Changing the presentation}

\begin{frame}
empty
\end{frame}

\subsection{Tietze transformations}

\begin{frame}
empty
\end{frame}

\section{Abelian quotients}

\subsection{Theory}

\begin{frame}
empty
\end{frame}

\subsection{Practice}

\begin{frame}
empty
\end{frame}

\end{document}


