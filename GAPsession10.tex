\documentclass[12pt]{article}

\pagestyle{empty}
\parindent0pt
%\usepackage{charter}
\usepackage{times}
\usepackage{amsfonts}
\usepackage[a4paper,margin=1in]{geometry}
\newcommand{\GAP}{\textsf{GAP}}
\newcommand{\GL}{\mathrm{GL}}
\newcommand{\Sp}{\mathrm{Sp}}
\newcommand{\Sym}{\mathrm{Sym}}
\newcommand{\ol}{\overline}
\newcommand{\ra}{\rightarrow}
\newcommand{\ms}{\mapsto}
\newcommand{\ti}{\tilde}
\newcommand{\bi}{\begin{itemize}}
\newcommand{\ei}{\end{itemize}}
\newcommand{\F}{\mathbb{F}}
\usepackage{hyperref}

\begin{document}
\begin{center}
\large LMS Short Course on Computational Group Theory

\Large Lab session 10

\large Different group representations \GAP
\end{center}

\begin{enumerate} 
\item In this exercise you will write two functions \texttt{MatRepr(G,p)}
and \texttt{PermRepr(G)}:
\begin{itemize}
\item[a)]
Let $G \le \mathrm{Sym}(n)$ and $B = \{ b_1, \ldots, b_n\}$ be the
standard basis of $V = \F_p^n$. Then $G$ acts on $B$ by $b_i^g =
b_{i^g}$ and one gets $\varphi : G \le \GL(n,p)$. The function
\texttt{MatRepr} shall compute $\varphi(G)$ for the inputs $G$ and $p$.
\item[b)]
Let $G \le \GL(n,p)$ and $V = \F_p^n$. Then $G$ acts on the elements
of $V$ by right multiplicaiton and one gets
$\psi : G \le \mathrm{Sym}(p^n)$. The function \texttt{PermRepr} shall 
compute $\psi(G)$ the input $G$.
\end{itemize}

\item 
Let $G = \Sp(4,3) \le \GL(4,3)$.
\begin{itemize}
\item[a)]
Fetch this group from the library of \GAP.
\item[b)]
Determine the permutation representation $\ol{G}$ of $G$ on $\F_3^4$.
\item[c)]
Determine the matrix representation $\ti{G}$ of $\overline{G}$ over
$\F_5$.
\item[d)]
Investigate $V = \F_5^{81}$ as module for $\ti{G}$.
\end{itemize}

\item
Define the groups $G$ and $H$ by
\begin{eqnarray*}
G &=& \left<
\left[ \begin{array}{rrr} 1&0&0 \\ 0&0&1 \\ -1&-1&-1 \end{array} \right],
\left[ \begin{array}{rrr} 0&0&-1 \\ 1&1&1 \\ 0&-1&0 \end{array} \right]
\right>, \\
H &=& \left<
\left[ \begin{array}{rr} -1&0 \\ 0&1 \end{array} \right],
\left[ \begin{array}{rr} -1&1 \\ 0&1\end{array} \right]
\right>.
\end{eqnarray*}
\begin{itemize}
\item[a)]
Which orders do the generators of $G$ and $H$ have?
\item[b)]
Are $G$ resp.~$H$ finite or infinite?
\item[c)]
If finite: identify the group.
\item[d)]
If infinite: how could you investigate the structure of the group
further?
\end{itemize}

\item Use \GAP\ to find the 12 smallest non-abelian finite simple
groups.
\end{enumerate}

\end{document}
