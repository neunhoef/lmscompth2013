%\documentclass[handout]{beamer}
\documentclass{beamer}

\mode<presentation>
{
  %\usetheme{Warsaw}
  %\usetheme{Antibes}
  %\usetheme{Berkeley}
  %\usetheme{Copenhagen}
  %\usetheme{Hannover}
  %\usetheme{JuanLesPins}
  %\usetheme[left]{Marburg}
  %\usetheme{PaloAlto}
  %\usetheme{sidebar}
  \usetheme{CambridgeUS}
  %\useinnertheme[shadow=true]{rounded}
  %\usetheme{Singapore}
  %\usecolortheme{crane}
  %\usecolortheme{lily}
  %\usecolortheme{beetle}
  \usecolortheme{orchid}
  % oder ...
  %\setbeamercovered{transparent}
  % oder auch nicht
  %\setbeamercovered{transparent}
  % or whatever (possibly just delete it)
  \setbeamertemplate{navigation symbols}{}
  %\setbeamertemplate{blocks}[rounded][shadow=true]
  %\setbeamercolor{block body}{bg=bg=normal text.bg!70!black}
  %\setbeamercolor{subsection in sidebar}{fg=white}
}


\usepackage[british]{babel}
% oder was auch immer
\usepackage{amsmath}
\usepackage{amssymb}
\usepackage[noend]{algorithmic}
%\usepackage{algorithm}
\usepackage{stmaryrd}

\theoremstyle{definition}
%\newtheorem{lemma}[theorem]{Lemma}
\newtheorem{defprop}[theorem]{Definition/Proposition}
\newtheorem{question}[theorem]{Question}
\newtheorem{assumption}[theorem]{Assumption}
\newtheorem{construction}[theorem]{Construction}
\newtheorem{principle}[theorem]{Principle}
\newtheorem{idea}[theorem]{Idea}
\newtheorem{proposition}[theorem]{Proposition}
\newtheorem{remark}[theorem]{Remark}
%\newtheorem{problem}[theorem]{Problem}
\usepackage[latin1]{inputenc}
% oder was auch immer

\usepackage{times}
\usepackage[T1]{fontenc}
% Oder was auch immer. Zu beachten ist, das Font und Encoding passen
% m�ssen. Falls T1 nicht funktioniert, kann man versuchen, die Zeile
% mit fontenc zu l�schen.

%\usepackage[mtbold,subscriptcorrection,mtpluscal]{mathtime}

%\usepackage{graphicx}
%\usepackage{rotating}

\usepackage[all]{xy}

\newcommand{\Struc}[1]{{\color{structure}#1}}
\newcommand{\Alert}[1]{{\color{alert}#1}}

\newcommand{\F}{\mathbb{F}}
\newcommand{\N}{\mathbb{N}}
\newcommand{\Z}{\mathbb{Z}}
\newcommand{\R}{\mathbb{R}}
\newcommand{\Oh}{\mathcal{O}}
\newcommand{\Aut}{\mathsf{Aut}}
\newcommand{\Stab}{\mathsf{Stab}}
\newcommand{\ob}{\mathsf{Ob}}
\newcommand{\mor}{\mathsf{Mor}}
\newcommand{\PSL}{\mathsf{PSL}}
\newcommand{\cR}{\mathcal{R}}
\newcommand\cyclic\circlearrowleft

\newcommand{\mybar}[1]{\overline{\raisebox{1.2ex}{}#1}}
\newcommand{\mybaremp}{\mybar{\ \ }}

\usepackage{calc}

\newsavebox{\linksrausbox}
\newlength{\linksrauslen}
\newcommand{\linksraus}[1]{\sbox{\linksrausbox}{#1}%
\settowidth{\linksrauslen}{\usebox{\linksrausbox}\ }%
\usebox{\linksrausbox} \begin{minipage}[t]{\textwidth-\linksrauslen}}
\newcommand{\linksrausend}{\end{minipage}\par}%\smallskip}

\usepackage{pgf,pgfarrows,pgfnodes}

%\pgfdeclareimage[width=1mm]{checkmark}{checkmark}
%\newcommand{\eofr}[1]{\vfill\vspace*{-#1mm}\hfill\pgfuseimage{checkmark}
%\par\vspace*{#1mm}\vspace*{-4mm}}

\newcommand{\GAP}{\textsf{GAP}}

\pgfdeclareimage[width=0.5in]{univstandlogo}{univstandlogo}
%\title[Computing the 2-modular characters of Fi$_{23}$] 
%          % (optional, nur bei langen Titeln n�tig)
%          {Computing the 2-modular characters of Fi$_{23}$}
\title[Finitely presented groups 3]
{Finitely presented groups 3}

\author% (optional, nur bei vielen Autoren)
{Max Neunh�ffer}

\institute[University of St Andrews] % (optional, aber oft n�tig)
{ 
\pgfuseimage{univstandlogo} \\[5mm]
%University of St Andrews  \\[2mm]
}

\date[1 August 2013] % (optional, sollte der abgek�rzte Konferenzname sein)
{LMS Short Course on Computational Group Theory \\ 29 July -- 2 August 2013}

\begin{document}
%\newcmykcolor{MyRedViolet}{0.07 0.90 0 0.34}
\begin{frame}
  \titlepage
\end{frame}

\section{Presentations for subgroups of FP groups}

\subsection{Finite index subgroups}

\begin{frame}
Let $G = \left< X \mid R \right>$, so that $G \cong F/N$ where
$F=F(X)$ and $N = \left<\left< R \right>\right>$.

\pause
We denote the natural map $\mybaremp :F \to G$ by bars.

\pause
Let $H = E/N < G$ and let $T \subseteq F$ be
a \Alert{right transversal} of $E$ in $F$:
\[ F = \bigcup_{t \in T}^{.} Et 
   \quad \mbox{ and thus } \quad
   G = \bigcup_{t \in T}^{.} H\mybar{t} \]

We assume $1_F \in T$ and represent elements of $F$ by \Struc{reduced
words}.

\pause
For a $w\in F$ define $\underline{w} :=
t \in T$ with $w \in Et$.

\pause
\begin{lemma}[Schreier (see Alexander's talk)]
The following set generates $E$:

\vspace*{-3mm}
\[ Y := \left\{ tx(\underline{tx})^{-1}
   \mid t \in T, x \in X, tx \neq \underline{tx} \right\} 
\subseteq F\setminus\{1_F\}  \]

\pause
\vspace*{-3mm}
Similarly, $H \le G$ is generated by the images $\mybar{Y} \subseteq G$.
\end{lemma}

\pause
\begin{theorem}[Nielsen-Schreier]
If $T$ is \Struc{prefix-closed}, then $E$ is a free group on $Y$.
\end{theorem}
\end{frame}

\subsection{The Reidemeister-Schreier method}

\begin{frame}

\Struc{Recall:}
\[ Y := \left\{ tx(\underline{tx})^{-1}
   \mid t \in T, x \in X, tx \neq \underline{tx} \right\} 
\subseteq F\setminus\{1_F\}  \]

\Struc{Idea of proofs}: How do we map $E$ to $F(Y)$?

\bigskip
\pause
Let $w = x_1x_2\cdots x_k \in E$ and set $t_i := \underline{x_1\cdots
x_i}$ for $0 \le i \le k$. 

\pause
Note $t_0 = t_k = 1_F$. \pause Then

\vspace*{-3mm}
\[ w = (t_0 x_1 t_1^{-1}) (t_1 x_2 t_2^{-1}) \cdots (t_{k-1} x_k
t_k^{-1}) \]
and all $t_{i-1} x_i t_i^{-1}$ are either $1_F$ or in $Y$.

\pause
\begin{itemize}
\item If $tx \in T$, then $tx(\underline{tx})^{-1}=1_F$. \pause 
\item Thus, if $T$ is \Struc{prefix-closed} and $w \in T$, all factors
are $1_F$.
\pause
\item Furthermore, for $w = tx \notin T$, all but the last factor are
$1_F$.
\pause
\end{itemize}
This implies that we
get a \Alert{well-defined isomorphism $\rho:E \to F(Y)$}.

\pause
\bigskip
\centerline{Assume from now on that $T$ is \Struc{prefix-closed}.}

\end{frame}

\begin{frame}
\begin{theorem}[Reidemeister-Schreier]
For $G=F/\left<\left<R\right>\right>$, $H=E/\left<\left<R\right>\right>$,
$T$ and $Y$ as above,
if $T$ is \Struc{prefix-closed}, then
$H=E/N$ is isomorphic to

\vspace*{-3mm}
\[ H' := \left< Y \mid \rho(twt^{-1}) \mbox{ for all } t \in T, w \in R
\right\}.
\]
\end{theorem}

\pause
Thus: if $[G:H] < \infty$ and we have a coset table for $G$ and
$H$, we can \Struc{compute the Schreier generators} and \Struc{write down this
presentation} for $H$ explicitly. This is the \Alert{Reidemeister-Schreier
Algorithm}.

\pause
\begin{example}[The dihedral group of order $8$]
Let $G := \left< s,t \mid s^4, t^2, stst\right>$ and $H := \left< s^2,t
\right>$. We know that $|G|=8$ and $H$ is a Klein four group.

\pause
\vspace*{-3mm}
Here is the \Struc{coset table}: 
\begin{tabular}{|l||c|c|c|}
\hline
   & $s$ & $s^{-1}$ & t \\
 \hline
 \hline
$1=H$  &  $2$  &  $ 2$    & $1$ \\
\hline
$2=Hs$ &  $1$  &  $ 1$    & $2$ \\
\hline
\end{tabular}

\pause
The {\color{blue}transversal $T$} is \Alert{$\{1,s\}$}, \pause 
the {\color{blue}Schreier generators} are:
$\{1ss^{-1}=1, 1t1^{-1}=t, ss1^{-1}=s^2, sts^{-1}\}\setminus\{1\}
\pause
 = \Alert{\{ t,s^2,sts^{-1} \}}$.
\end{example}
\end{frame}

\begin{frame}
\begin{example}[The dihedral group of order $8$ (continued)]
Let $G := \left< s,t \mid s^4, t^2, stst\right>$ and $H := \left< s^2,t
\right>$.

{\color{blue}Transversal:} $T=\{1,s\}$, 
{\color{blue}Schreier generators:} $\{ t,s^2,sts^{-1} \}$.

\pause
If $F:=F(s,t)$ and $N := \left<\left< s^4,t^2,stst\right>\right>$
then $G=F/N$ and $E/N:=H$, \pause 

thus $E$ is {\color{red}free on 
$(A,B,C) := (t,s^2,sts^{-1})$} (it is of index $2$ in $F$).

\pause
{\color{blue}Reidemeister-Schreier} now gives the following relators:

\begin{itemize}
\item $\rho(1s^41^{-1}) = \rho(s^4) =
(1{\color{blue}s}s^{-1})(s{\color{blue}s}1^{-1})(1{\color{blue}s}s^{-1})(s{\color{blue}s}1^{-1})=B^2$,
\pause
\item $\rho(ss^4s^{-1} = \rho(s^4) = B^2$,
\pause
\item $\rho(1t^21^{-1}) = \rho(t^2) =
(1{\color{blue}t}1^{-1})(1{\color{blue}t}1^{-1}) 
= A^2$,
\pause
\item $\rho(st^2s^{-1}) = \rho(st^2s^{-1}) 
=
(1{\color{blue}s}s^{-1})(s{\color{blue}t}s^{-1})(s{\color{blue}t}s^{-1})(s{\color{blue}s^{-1}}1^{-1}) 
= C^2$,
\pause
\item $\rho(1stst1^{-1}) = \rho(stst) 
=
(1{\color{blue}s}s^{-1})(s{\color{blue}t}s^{-1})(s{\color{blue}s}1^{-1})(1{\color{blue}t}1^{-1}) 
= CBA$,
\pause
\item $\rho(sststs^{-1}) 
=
(1{\color{blue}s}s^{-1})(s{\color{blue}s}1^{-1})(1{\color{blue}t}1^{-1})
(1{\color{blue}s}s^{-1})(s{\color{blue}t}s^{-1})(s{\color{blue}s^{-1}}1^{-1})
= BAC$.
\end{itemize}
\pause
Thus, we get that $H \cong \left< A,B,C \mid B^2, A^2, C^2, CBA, BAC\right>$.
\end{example}

\end{frame}

\begin{frame}
Finding presentations on the \Struc{user supplied generators} works similarly.

\vspace*{2cm}
\pause
\centerline{\url{http://tinyurl.com/MNGAPsess/GAP\_FP\_6.g}}
% GAP session here to type do the same example and then a bigger one.
\end{frame}

\section{Proving groups to be infinite}

\subsection{Combine low index with Abelian invariants}

\begin{frame}
\begin{problem}
Let $G := \left< X \mid R \right>$. How could we prove that $|G| = \infty$?
\end{problem}

\pause
\textbf{First idea:} Abelian invariants, but what if they are all positive?

\bigskip
\pause

\textbf{Second idea:}
\begin{itemize}
\item Compute some \Struc{low index subgroups}, result is a coset table
\pause
\item Use \Struc{Reidemeister-Schreier} to find \Alert{presentations
for them}.
\pause
\item Compute the \Struc{Abelian invariants} on these presentations.
\pause
\item If we find a $0$, the group $G$ is infinite as well.
\end{itemize}

\pause
\vspace*{5mm}
\centerline{\url{http://tinyurl.com/MNGAPsess/GAP\_FP\_7.g}}
% The modular group and maybe something else.
\end{frame}

\section{Rewrite systems}

\subsection{Fundamental definitions}

\begin{frame}
\begin{definition}[Rewrite system]
Let $A$ be a finite alphabet and $A^*$ the set of all words in $A$.

A \Struc{rewrite system (RWS)} is a set of \Alert{rules}
$v \to w$ where $v,w \in A^*$.

\pause
We then say that $avb \to awb$ for all $a,b \in A^*$ \pause and write
$c \Rightarrow d$, if $c=d$ or there is a finite tuple $(c_1,c_2,\ldots,c_k)$
of words with 

\vspace*{-3mm}
\[ c \to c_1 \to c_2 \to \cdots \to c_k \to d\] 

%\vspace*{-3mm}
%\pause
%That is, $\Rightarrow$ is the \Struc{transitive and reflexive closure} of the
%relation $\to$ on $A^*$.
\pause
A word $v \in A^*$ is called \Struc{irreducible}, if there is no $w
\in A^*$ with $v \to w$.
\end{definition}

\pause
\begin{definition}[Termination]
A RWS is called \Struc{terminating}, if
there is no infinite chain of words
$w_1 \to w_2 \to w_3 \to \cdots$
\end{definition}

\pause
\begin{definition}[Confluence and completeness]
A RWS is called \Struc{confluent}, if for all $a,b,c \in A^*$
with $a \Rightarrow b$ and $a \Rightarrow c$ there is a $d \in A^*$
with $b \Rightarrow d$ and $c \Rightarrow d$.

\pause
A \Struc{terminating} and \Struc{confluent} RWS is called
\Struc{complete}.
\end{definition}
\end{frame}

\subsection{Local confluence}

\begin{frame}
\begin{lemma}
If a RWS is terminating, then every word $w \in A^*$ can only be
rewritten to finitely many words.
\end{lemma}

\pause
\begin{defprop}[Local confluence]
A RWS is called \Struc{locally confluent}, if for all $a,b,c \in A^*$
with $a \to b$ and $a \to c$ there is a $d \in A^*$
with $b \Rightarrow d$ and $c \Rightarrow d$.

\pause
A \Alert{terminating} and \Alert{locally confluent} RWS is \Alert{complete}.
\end{defprop}

\pause
\begin{definition}[Equivalence]
Let $\Leftrightarrow$ be the \Struc{transitive, reflexive and
symmetric closure} of $\to$, i.e., the finest equivalence relation
with $v \Leftrightarrow w$ for all rules $v \to w$.
\end{definition}

\pause
\begin{lemma}
If a RWS is complete, then every $\Leftrightarrow$ class contains
\Struc{exactly one irreducible element} and all words in the class 
\Struc{can be rewritten} to it.
\end{lemma}
\end{frame}

\subsection{Critical pairs}

\begin{frame}
\begin{question}
How can it ever happen, that \Struc{$a \to b$} and \Struc{$a \to c$}, 
but that $b$ and $c$ \Alert{cannot be rewritten to any common word $d$}?
\end{question}

\pause
Assume $v_1\to w_1$ and $v_2 \to w_2$ are rules, if $a=p v_1 q v_2
r$, then both rules apply, but we have:
\[ \xymatrix{ & p v_1 q v_2 r \only<3->{\ar[dl] \ar[dr]} & \\
             \uncover<3->{p w_1 q v_2 r} \only<4->{\ar[dr]} && 
             \uncover<3->{p v_1 q w_2 r} \only<4->{\ar[dl]} \\
             & \uncover<4->{p w_1 q w_2 r}&} \]
             \vspace*{3mm}
\uncover<5->{%
\centerline{Thus: \Struc{the left hand sides have to} \Alert{overlap}!}}

\end{frame}

\begin{frame}
\begin{definition}[Critical pair]
A pair of rules $v_1 \to w_1$ and $v_2 \to w_2$ is called a \Struc{critical
pair}, if:
\begin{itemize}
\item $v_1 = r\Alert{s}$ and $v_2 = \Alert{s}t$ for some $r,s,t \in A^*$, or
\item $v_1 = r\Alert{s}t$ and $v_2 = \Alert{s}$ for some $r,s,t \in A^*$,
\end{itemize}
with $s \neq \varepsilon$ in both cases.
\end{definition}

\pause
\begin{lemma}
A RWS is \Struc{locally confluent} if and only if the following
conditions are fulfilled \Alert{for all critical pairs} 
$v_1 \to w_1$ and $v_2 \to w_2$:
\begin{itemize}
\item If $v_1=rs$ and $v_2=st$, then $\exists w \in A^*$ with 
$w_1 t \Rightarrow w$ and $r w_2 \Rightarrow w$.
\item If $v_1=rst$ and $v_2=s$ then $\exists w \in A^*$ with
$rw_2t \Rightarrow w$ and $w_1 \Rightarrow w$.
\end{itemize}
\end{lemma}

\bigskip
\pause
We can \Struc{check confluence} of a \Struc{finite,
terminating RWS} \Alert{algorithmically}!
\end{frame}

\section{The Knuth-Bendix procedure}

\subsection{The idea}

\begin{frame}
\begin{definition}[Reduction ordering]
A well-ordering on $A^*$ is called a \Struc{reduction ordering}, if $u \le
v$ implies $uw \le vw$ and $wu \le wv$ for all $u,v,w \in A^*$.

\pause
\textbf{Example:} ``shortlex'': sort first by length and then
lexicographically.
\end{definition}

\pause
\begin{block}{Idea of the Knuth-Bendix completion procedure}
Start with a \Struc{finite RWS} and choose a \Struc{reduction ordering}
such that $v > w$ for all rules $v \to w$.

\pause
Consider all possible \Struc{critical pairs} $rs \to w_1$ and
$st \to w_2$, and:
\begin{itemize}
\item \Struc{rewrite} $w_1t \Rightarrow w'_1$ and $rw_2 \Rightarrow w'_2$ with
$w'_1$ and $w'_2$ \Alert{irreducible},
\item if $w'_1 \neq w'_2$, then 
\begin{itemize}
\item \mbox{}\rlap{\Struc{either add}}\hspace*{2cm}% 
$w'_1 \to w'_2$ if $w'_1 > w'_2$,
\item \mbox{}\rlap{\Struc{or add}}\hspace*{2cm}% 
$w'_2 \to w'_1$ if $w'_2 > w'_1$.
\end{itemize}
\end{itemize}
\pause
(and similarly for $rst \to w_1$ and $s \to w_2$)
\end{block}

\smallskip
\pause
If the RWS is already or becomes confluent this procedure terminates.
\end{frame}

\subsection{Properties of the procedure}

\begin{frame}
Every \Struc{minimal word} (in its class) is \Struc{irreducible}.

\pause
If the RWS is \Struc{confluent}, then the converse is true, too.

\pause
\begin{proposition}
If $v \Leftrightarrow w$ and $v > w$, then after running Knuth-Bendix
\Alert{long enough}, we will get $v \Rightarrow w$.
\end{proposition}

\pause
\begin{proposition}
If the RWS has only \Struc{finitely many $\Leftrightarrow$-classes}, then
Knuth-Bendix \Alert{will terminate} with a complete RWS.
\end{proposition}

\pause
\begin{remark}
With a complete RWS we have a good way to decide $v \Leftrightarrow w$.
\end{remark}
\end{frame}

\subsection{Knuth-Bendix and groups}

\begin{frame}
Let $G = \left< X \mid R\right>$. \pause

\bigskip
We can present $G$ as a \Struc{monoid} by adding relators
$xx^{-1}=\varepsilon$ and
$x^{-1}x=\varepsilon$ for all $x \in X$. \pause

\bigskip
For $\hat X := X \cup X^{-1}$ the set $\hat X^*$ is
a \Struc{free monoid} and $G$ is a quotient. \pause

\bigskip
If we \Struc{choose a reduction order} and 
add \Alert{one rule for each equation} in the
presentation, then $\Leftrightarrow$ is precisely the congruence on
$\hat X^*$ such that $G \cong \hat X^*/\Leftrightarrow$.

\pause
\bigskip
Therefore, completing this RWS \Struc{solves the word problem in $G$}.

\pause
\vspace*{5mm}
\centerline{\url{http://tinyurl.com/MNGAPsess/GAP\_FP\_8.g}}
% Show a successful Knuth-Bendix run with KBMAG

\end{frame}

\end{document}

