%\documentclass[handout]{beamer}
\documentclass{beamer}

\mode<presentation>
{
  %\usetheme{Warsaw}
  %\usetheme{Antibes}
  %\usetheme{Berkeley}
  %\usetheme{Copenhagen}
  %\usetheme{Hannover}
  %\usetheme{JuanLesPins}
  %\usetheme[left]{Marburg}
  %\usetheme{PaloAlto}
  %\usetheme{sidebar}
  \usetheme{CambridgeUS}
  %\useinnertheme[shadow=true]{rounded}
  %\usetheme{Singapore}
  %\usecolortheme{crane}
  %\usecolortheme{lily}
  %\usecolortheme{beetle}
  \usecolortheme{orchid}
  % oder ...
  %\setbeamercovered{transparent}
  % oder auch nicht
  %\setbeamercovered{transparent}
  % or whatever (possibly just delete it)
  \setbeamertemplate{navigation symbols}{}
  %\setbeamertemplate{blocks}[rounded][shadow=true]
  %\setbeamercolor{block body}{bg=bg=normal text.bg!70!black}
  %\setbeamercolor{subsection in sidebar}{fg=white}
}


\usepackage[british]{babel}
% oder was auch immer
\usepackage{amsmath}
\usepackage{amssymb}
\usepackage[noend]{algorithmic}
%\usepackage{algorithm}
\usepackage{stmaryrd}

\theoremstyle{definition}
%\newtheorem{lemma}[theorem]{Lemma}
\newtheorem{defprop}[theorem]{Definition/Proposition}
\newtheorem{question}[theorem]{Question}
\newtheorem{assumption}[theorem]{Assumption}
\newtheorem{construction}[theorem]{Construction}
\newtheorem{principle}[theorem]{Principle}
\newtheorem{idea}[theorem]{Idea}
\newtheorem{proposition}[theorem]{Proposition}
%\newtheorem{problem}[theorem]{Problem}
\usepackage[latin1]{inputenc}
% oder was auch immer

\usepackage{times}
\usepackage[T1]{fontenc}
% Oder was auch immer. Zu beachten ist, das Font und Encoding passen
% m�ssen. Falls T1 nicht funktioniert, kann man versuchen, die Zeile
% mit fontenc zu l�schen.

%\usepackage[mtbold,subscriptcorrection,mtpluscal]{mathtime}

%\usepackage{graphicx}
%\usepackage{rotating}

\usepackage[all]{xy}

\newcommand{\Struc}[1]{{\color{structure}#1}}
\newcommand{\Alert}[1]{{\color{alert}#1}}

\newcommand{\F}{\mathbb{F}}
\newcommand{\N}{\mathbb{N}}
\newcommand{\Z}{\mathbb{Z}}
\newcommand{\R}{\mathbb{R}}
\newcommand{\Oh}{\mathcal{O}}
\newcommand{\Aut}{\mathsf{Aut}}
\newcommand{\Stab}{\mathsf{Stab}}
\newcommand{\ob}{\mathsf{Ob}}
\newcommand{\mor}{\mathsf{Mor}}
\newcommand{\PSL}{\mathsf{PSL}}
\newcommand{\cR}{\mathcal{R}}
\newcommand\cyclic\circlearrowleft

\newcommand{\mybar}[1]{\overline{\raisebox{1.2ex}{}#1}}
\newcommand{\mybaremp}{\mybar{\ \ }}

\usepackage{calc}

\newsavebox{\linksrausbox}
\newlength{\linksrauslen}
\newcommand{\linksraus}[1]{\sbox{\linksrausbox}{#1}%
\settowidth{\linksrauslen}{\usebox{\linksrausbox}\ }%
\usebox{\linksrausbox} \begin{minipage}[t]{\textwidth-\linksrauslen}}
\newcommand{\linksrausend}{\end{minipage}\par}%\smallskip}

\usepackage{pgf,pgfarrows,pgfnodes}

%\pgfdeclareimage[width=1mm]{checkmark}{checkmark}
%\newcommand{\eofr}[1]{\vfill\vspace*{-#1mm}\hfill\pgfuseimage{checkmark}
%\par\vspace*{#1mm}\vspace*{-4mm}}

\newcommand{\GAP}{\textsf{GAP}}

\pgfdeclareimage[width=0.5in]{univstandlogo}{univstandlogo}
%\title[Computing the 2-modular characters of Fi$_{23}$] 
%          % (optional, nur bei langen Titeln n�tig)
%          {Computing the 2-modular characters of Fi$_{23}$}
\title[Finitely presented groups 3]
{Finitely presented groups 3}

\author% (optional, nur bei vielen Autoren)
{Max Neunh�ffer}

\institute[University of St Andrews] % (optional, aber oft n�tig)
{ 
\pgfuseimage{univstandlogo} \\[5mm]
%University of St Andrews  \\[2mm]
}

\date[1 August 2013] % (optional, sollte der abgek�rzte Konferenzname sein)
{LMS Short Course on Computational Group Theory \\ 29 July -- 2 August 2013}

\begin{document}
%\newcmykcolor{MyRedViolet}{0.07 0.90 0 0.34}
\begin{frame}
  \titlepage
\end{frame}

\section{Presentations for subgroups of FP groups}

\subsection{Finite index subgroups}

\begin{frame}
Let $G = \left< X \mid R \right>$, so that $G \cong F/N$ where
$F=F(X)$ and $N = \left<\left< R \right>\right>$.

\pause
We denote the natural map $\mybaremp :F \to G$ by bars.

\pause
Let $H = E/N < G$ and let $T \subseteq F$ be
a \Alert{right transversal} of $E$ in $F$:
\[ F = \bigcup_{t \in T}^{.} Et 
   \quad \mbox{ and thus } \quad
   G = \bigcup_{t \in T}^{.} H\mybar{t} \]

We assume $1_F \in T$ and represent elements of $F$ by \Struc{reduced
words}.

\pause
For a $w\in F$ define $\underline{w} :=
t \in T$ with $w \in Et$.

\pause
\begin{lemma}[Schreier (see Alexander's talk)]
The following set generates $E$:

\vspace*{-3mm}
\[ Y := \left\{ tx(\underline{tx})^{-1}
   \mid t \in T, x \in X, tx \neq \underline{tx} \right\} 
\subseteq F\setminus\{1_F\}  \]

\pause
\vspace*{-3mm}
Similarly, $H \le G$ is generated by the images $\mybar{Y} \subseteq G$.
\end{lemma}

\pause
\begin{theorem}[Nielsen-Schreier]
If $T$ is \Struc{prefix-closed}, then $E$ is a free group on $Y$.
\end{theorem}
\end{frame}

\subsection{The Reidemeister-Schreier method}

\begin{frame}

\Struc{Recall:}
\[ Y := \left\{ tx(\underline{tx})^{-1}
   \mid t \in T, x \in X, tx \neq \underline{tx} \right\} 
\subseteq F\setminus\{1_F\}  \]

\Struc{Idea of proofs}: How do we map $E$ to $F(Y)$?

\bigskip
\pause
Let $w = x_1x_2\cdots x_k \in E$ and set $t_i := \underline{x_1\cdots
x_i}$ for $0 \le i \le k$. 

\pause
Note $t_0 = t_k = 1_F$. \pause Then

\vspace*{-3mm}
\[ w = (t_0 x_1 t_1^{-1}) (t_1 x_2 t_2^{-1}) \cdots (t_{k-1} x_k
t_k^{-1}) \]
and all $t_{i-1} x_i t_i^{-1}$ are either $1_F$ or in $Y$.

\pause
\begin{itemize}
\item If $tx \in T$, then $tx(\underline{tx})^{-1}=1_F$. \pause 
\item Thus, if $T$ is \Struc{prefix-closed} and $w \in T$, all factors
are $1_F$.
\pause
\item Furthermore, for $w = tx \notin T$, all but the last factor are
$1_F$.
\pause
\end{itemize}
This implies that we
get a \Alert{well-defined isomorphism $\rho:E \to F(Y)$}.

\pause
\bigskip
\centerline{Assume from now on that $T$ is \Struc{prefix-closed}.}

\end{frame}

\begin{frame}
\begin{theorem}[Reidemeister-Schreier]
For $G=F/\left<\left<R\right>\right>$, $H=E/\left<\left<R\right>\right>$,
$T$ and $Y$ as above,
if $T$ is \Struc{prefix-closed}, then
$H=E/N$ is isomorphic to

\vspace*{-3mm}
\[ H' := \left< Y \mid \rho(twt^{-1}) \mbox{ for all } t \in T, w \in R \right\}.
\]
\end{theorem}

\end{frame}
\section{Rewrite systems}

\subsection{Fundamental definitions}

\subsection{Local confluence and critical pairs}

\section{The Knuth-Bendix procedure}

\subsection{Completing rewrite systems}

\end{document}

