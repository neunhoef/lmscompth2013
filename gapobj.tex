%\documentclass[handout]{beamer}
\documentclass{beamer}

\mode<presentation>
{
  %\usetheme{Warsaw}
  %\usetheme{Antibes}
  %\usetheme{Berkeley}
  %\usetheme{Copenhagen}
  %\usetheme{Hannover}
  %\usetheme{JuanLesPins}
  %\usetheme[left]{Marburg}
  %\useinnertheme[shadow=true]{rounded}
  \usetheme{CambridgeUS}
  %\usetheme{PaloAlto}
  %\usetheme{sidebar}
  %\usecolortheme{crane}
  %\usecolortheme{lily}
  %\usecolortheme{beetle}
  \usecolortheme{orchid}
  % oder ...
  %\setbeamercovered{transparent}
  % oder auch nicht
  %\setbeamercovered{transparent}
  % or whatever (possibly just delete it)
  \setbeamertemplate{navigation symbols}{}
  %\setbeamertemplate{blocks}[rounded][shadow=true]
  %\setbeamercolor{block body}{bg=bg=normal text.bg!70!black}
  %\setbeamercolor{subsection in sidebar}{fg=white}
}


\usepackage[british]{babel}
% oder was auch immer
\usepackage{amsmath}
\usepackage{amssymb}

\theoremstyle{definition}
\newtheorem{questions}[theorem]{Questions}
\usepackage[latin1]{inputenc}
% oder was auch immer

\usepackage{times}
\usepackage[T1]{fontenc}
% Oder was auch immer. Zu beachten ist, das Font und Encoding passen
% m�ssen. Falls T1 nicht funktioniert, kann man versuchen, die Zeile
% mit fontenc zu l�schen.

%\usepackage{graphicx}
%\usepackage{rotating}

\newcommand{\Struc}[1]{{\color{structure}#1}}
\newcommand{\Alert}[1]{{\color{alert}#1}}

\usepackage{calc}

\newsavebox{\linksrausbox}
\newlength{\linksrauslen}
\newcommand{\linksraus}[1]{\sbox{\linksrausbox}{#1}%
\settowidth{\linksrauslen}{\usebox{\linksrausbox}\ }%
\usebox{\linksrausbox} \begin{minipage}[t]{\textwidth-\linksrauslen}}
\newcommand{\linksrausend}{\end{minipage}\par}%\smallskip}

\newcommand{\GAP}{{\sf GAP}}

\title{Objects, types and method selection in \GAP}

\author% (optional, nur bei vielen Autoren)
{Max Neunh�ffer}

\institute % (optional, aber oft n�tig)
{ University of St Andrews}

\date[1.8.2013] % (optional, sollte der abgek�rzte Konferenzname sein)
{1.8.2013}

\begin{document}
%\newcmykcolor{MyRedViolet}{0.07 0.90 0 0.34}
\begin{frame}
  \titlepage
\end{frame}

%\begin{frame}
%  \frametitle{Overview}
%  \tableofcontents
%  % Die Option [pausesections] k�nnte n�tzlich sein.
%\end{frame}

\section{Objects, operations and method selection}

\subsection{The idea}

\begin{frame}
\frametitle{The idea}
\begin{center}
  {\GAP} objects represent mathematical objects.

  \pause
  \medskip

  There are \Struc{``operations''} and \Struc{``methods''}.

  \pause
  \medskip
  Properties of objects (their \Struc{type})
  
  $\Downarrow$

  Selection of the ``right'' method

  \pause
  \bigskip
  \Struc{Objects can ``learn'' during their lifetime}

  (i.e.~change their type)

  \pause
  \medskip
  \Alert{The methods used change as a consequence!}

  \pause
\end{center}
{\GAP} thus uses:
\begin{itemize}
\item dynamic typing at runtime
\pause
\item a static database of methods
\pause
\item ``just in time'' method selection
\end{itemize}
\end{frame}

\subsection{Types}

\begin{frame}
\frametitle{Types}
A \Struc{type in {\GAP}} is a pair:

\begin{center}
(a ``family'', a bit list of ``elementary filters'')
\end{center}

\pause
The \Struc{families} form a \Alert{partition of the set of objects}.

\hspace*{1cm} one part is for example the \texttt{PermutationsFamily}

\pause
\medskip
An \Struc{elementary filter} is \Alert{both}
\begin{itemize}
\item a bit in the 2nd component of the type \Alert{and}
\item the set of all objects, which have that bit set in their type.
\end{itemize}

\pause
\medskip
A \Struc{filter} is \Alert{either}
\begin{itemize}
\item an elementary filter \Alert{or}
\item an and-composition of elementary filters.
\end{itemize}

\pause
\medskip
Every object o ``is'' either ``in some given filter'' or not.

This can be tested with \texttt{FILTERNAME(o)}.

\pause
\medskip
Examples: \texttt{IsSolvable}, \texttt{IsNilpotent}, \texttt{IsAbelian}
\end{frame}

\subsection{Operations and methods}

\begin{frame}[fragile]
\frametitle{Operations and methods}

An \Struc{operation} is a collection of methods.

One declares
\begin{itemize}
\item the name,
\item the number of arguments, and
\item a filter for each argument.
\end{itemize}

\medskip
\pause
\verb!DeclareOperation("Size",[IsGroup]);!

\bigskip
\pause
One installs one or more \Struc{methods}:
\begin{itemize}
 \item These are functions with the right number of arguments.
 \item One can give further restrictions:
\end{itemize}

\pause
\begin{semiverbatim}
InstallMethod(Size,
\hspace*{5mm}[IsGroup and IsPermGroup],
\hspace*{5mm}function(p) ... return ...; end);
\end{semiverbatim}

\pause
We call these restrictions \Struc{``required filters''}.
\end{frame}

\begin{frame}
\frametitle{The method selection}

If somebody calls \texttt{Size(g)} for an object \texttt{g},

\medskip
\pause
\begin{itemize}
\item \GAP\ determines the \Struc{type} of \texttt{g},
\pause
\item considers \Struc{all} methods for \texttt{Size},
\pause
\item determines, which are \Struc{applicable} (are in all
required filters),
\pause
\item and calls \Struc{the method} that
\begin{itemize}
\item \Alert{is applicable}, and
\item \Alert{has the most required filters}

      (if two or more have the same required filters it takes the one
which was installed later).
\end{itemize}
\end{itemize}

\medskip
\pause
This only works efficiently by a very tricky \Struc{method cache}!

\bigskip
\pause
\linksraus{\Struc{More accurately:}} Each elementary filter has a ``rank''.

The method with the highest sum of ranks of the required filters is
chosen.
\linksrausend
\end{frame}

\section{Conventions}

\subsection{Families}

\begin{frame}
\frametitle{The idea behind families}

The \Struc{families} \Alert{partition} the set of all objects.

\smallskip
In contrast, the \Struc{filters} form a \Alert{hierarchy} of sets.

\medskip
\pause
e.g.: \texttt{PermutationsFamily}, \texttt{CyclotomicsFamily}.

\medskip
\pause
For FP groups all elements of one such group form a
family.

\bigskip
\pause
A \Struc{collection} consists of objects from \Alert{the same family}.

\medskip
\pause
One can form the ``CollectionsFamily'' of any family,

\smallskip
\pause
and the ``ElementsFamily'' of each CollectionsFamily:

\pause
\begin{semiverbatim}
\small
gap> f:=CollectionsFamily(CyclotomicsFamily);;

gap> CyclotomicsFamily=ElementsFamily(f);

true

gap> FamilyObj((1,2,3))=PermutationsFamily;

true
\end{semiverbatim}

\end{frame}

\subsection{Categories and representations}

\begin{frame}
\frametitle{Categories and representations}

``Categories'' and ``representations'' are nothing but
\Struc{elementary filters} with a bit of \Alert{philosophy} in the
background!

\medskip
\pause
Objects in the same \Struc{category} are 
\Struc{(mathematically) similar objects}.

Objects \Alert{never change category}!

\medskip
\pause
Mathematically similar or equal objects can be
\Struc{represented differently}, then they should lie in
\Alert{different representations}.

\bigskip
\pause
\texttt{IsPerm} is a category.

\medskip
\texttt{IsPerm2Rep} and \texttt{IsPerm4Rep} are representations.

\pause
\bigskip
\Struc{Categories} usually occur in \Alert{declarations of
operations},

\Struc{representations} usually occur as \Alert{required
filters} in method installations.
\end{frame}


\subsection{Inheritance}

\begin{frame}[fragile]
\frametitle{Inheritance in {\GAP}}

Inheritance works with \Struc{subfilters}.

\pause
\medskip
One declares subfilters and constructs objects that \Struc{lie in these
additional subfilters}.

\pause
\medskip
If one needs special methods, these are installed with the subfilters
as \Struc{additional requirements}.

\pause
\bigskip
\Struc{Hypothetical example:}

{\small
\begin{verbatim}
DeclareCategory("IsGroup",IsObject);
DeclareCategory("IsAbelianGroup",IsGroup);
DeclareOperation("Size",[IsGroup]);
InstallMethod(Size,"for arbitrary groups",
              [IsGroup],
              function(g) ... end);
InstallMethod(Size,"for abelian groups",
              [IsAbelianGroup],
              function(a) ... end);
\end{verbatim}
}

\end{frame}

\section{An example}

\subsection{The declarations}

\begin{frame}[fragile]
\frametitle{The declarations}

{\small
\begin{verbatim}
BindGlobal("BlubbsFamily",
           NewFamily("BlubbsFamily"));
DeclareCategory("IsBlubb",
                IsComponentObjectRep);
DeclareRepresentation("IsBlubbDenseRep",
                      IsBlubb,["wo","p"]);
BindGlobal("BlubbDenseType",
  NewType(BlubbsFamily,IsBlubbDenseRep));
\end{verbatim}
\pause
\begin{verbatim}
DeclareOperation("Blubb",[IsString,IsInt]);
DeclareOperation("IsShort",[IsBlubb]);
DeclareOperation("NrLetters",[IsBlubb]);
\end{verbatim}
\pause
\begin{verbatim}
InstallMethod(Blubb,"constructor",
  [IsString,IsInt], function(s,i)
    local r;
    r := rec(wo:=s,p:=i);
    return Objectify(BlubbDenseType,r);
  end);
\end{verbatim}
}

\end{frame}

\subsection{The implementations}

\begin{frame}[fragile]
\frametitle{The implementations}

{\small
\begin{verbatim}
InstallMethod(IsShort,"for dense Blubbs",
  [IsBlubbDenseRep],
  function(bl) 
    return Length(bl!.wo) <= 5;
  end);
\end{verbatim}
\pause
\begin{verbatim}
InstallMethod(NrLetters,"for dense Blubbs",
  [IsBlubbDenseRep],
  function(bl)
    return Length(Set(bl!.wo));
  end);
\end{verbatim}
\pause
\begin{verbatim}
InstallMethod(ViewObj,"for dense Blubbs",
  [IsBlubbDenseRep],
  function(bl)
    Print("<a dense blubb wo=",bl!.wo,
          " p=",bl!.p,">");
  end);
\end{verbatim}
}
\end{frame}

\subsection{Usage examples}

\begin{frame}[fragile]
\frametitle{Usage}

One can now use \texttt{Blubb}-objects as follows:

{\small
\begin{verbatim}
gap> b := Blubb("abac",17);
<a dense blubb wo=abac p=17>
gap> NrLetters(b);
3
gap> IsShort(b);
true
gap> b!.wo;
"abac"
gap> b!.p;
17
\end{verbatim}
}
\pause
One should install methods for
\begin{itemize}
\item \texttt{ViewObj} (for the user to see a \Struc{concise} description)
\item \texttt{PrintObj} (if possible \Struc{{\GAP}-parsable})
\item and possibly \texttt{Display} (\Struc{nicely formatted} description
for the user).
\end{itemize}
\end{frame}

\section{Further niceties}

\subsection{Properties}

\begin{frame}[fragile]
\frametitle{Properties}

A \Struc{Property} ``XYZ'' is realised by:
\begin{itemize}
\item an \Struc{elementary filter} \texttt{HasXYZ} and
\item an \Struc{elementary filter} \texttt{XYZ}.
\end{itemize}

\pause
\medskip
Properties are declared like this:

\hspace*{5mm}\verb|DeclareProperty("IsShort",IsBlubb);|

\pause
\medskip
This \Struc{automatically defines}
\begin{itemize}
\item an elementary filter \texttt{HasIsShort},
\pause
\item an elementary filter \texttt{IsShort},
\pause
\item an operation \texttt{IsShort},
\pause
\item a method for \texttt{IsShort} for objects in the filter
      \texttt{IsBlubb and HasIsShort}, which just checks the type, and
\pause
\item an operation with method \texttt{SetIsShort}.
\end{itemize}

\end{frame}

\subsection{Attributes}

\begin{frame}[fragile]
\frametitle{Attributes}

\hspace*{5mm}\verb!DeclareAttribute("NrLetters",IsBlubb);!

\pause
\medskip
defines \Struc{automatically}
\begin{itemize}
\item an elementary filter \texttt{HasXYZ},
\pause
\item an operation \texttt{XYZ}.
\end{itemize}

\pause
\Alert{If one inherits} from
\Struc{\texttt{IsComponentObjectRep and IsAttributeStoringRep}},
then one also gets:
\begin{itemize}
\item An operation \texttt{SetXYZ} for \verb![IsBlubb,IsObject]! that
stores the 2nd argument in the \texttt{!.XYZ}-component and sets
\texttt{HasXYZ}.
\pause
\item Every method for \texttt{XYZ} stores its result automatically in
that component and sets \texttt{HasXYZ}.
\pause
\item A very highly ranked method for \texttt{XYZ} for objects in the filter
\texttt{IsBlubb and HasXYZ} that simply returns \texttt{!.XYZ}.
\end{itemize}

\end{frame}

\begin{frame}[fragile]
%\frametitle{The example --- revisited}

In our example, we can simply replace
{\small
\begin{verbatim}
DeclareCategory("IsBlubb",
                IsComponentObjectRep);
DeclareOperation("IsShort",[IsBlubb]);
DeclareOperation("NrLetters",[IsBlubb]);
\end{verbatim}}
by
{\small
\begin{verbatim}
DeclareCategory("IsBlubb",
                IsAttributeStoringRep);
DeclareProperty("IsShort",IsBlubb);
DeclareAttribute("NrLetters",IsBlubb);
\end{verbatim}}
and automatically get caching:
\pause
{\small
\begin{verbatim}
gap> b := Blubb("abac",17);
<a dense blubb wo=abac p=17>
gap> HasNrLetters(b);
false
gap> NrLetters(b);;
gap> HasNrLetters(b);
true
\end{verbatim}
}
\end{frame}

\subsection{Debugging}

\begin{frame}[fragile]
\frametitle{Debugging}
If you want to see which methods are available:

{\small
\begin{verbatim}
gap> ApplicableMethod(NrLetters,[b],3,"all");
#I  Searching Method for NrLetters with 1 \
                               arguments:
#I  Total: 2 entries
#I  Method 1: ``NrLetters: system getter'',\
                        value: 2*SUM_FLAGS+4
#I   - 1st argument needs \
                 [ "IsAttributeStoringRep",\
                   "Tester(NrLetters)" ]
#I  Method 2: ``NrLetters: for dense \
                          Blubbs'', value: 3
#I  Skipped:
[ function( bl ) ... end ]
\end{verbatim}}

\end{frame}

\begin{frame}[fragile]
\frametitle{The complete example}
{\small
\begin{verbatim}
BindGlobal("BlubbsFamily",
           NewFamily("BlubbsFamily"));
DeclareCategory("IsBlubb",
                IsAttributeStoringRep);
DeclareRepresentation("IsBlubbDenseRep",
                      IsBlubb,["wo","p"]);
BindGlobal("BlubbDenseType",
  NewType(BlubbsFamily,IsBlubbDenseRep));
\end{verbatim}

\begin{verbatim}
DeclareOperation("Blubb",[IsString,IsInt]);
DeclareProperty("IsShort",IsBlubb);
DeclareAttribute("NrLetters",IsBlubb);
\end{verbatim}

\begin{verbatim}
InstallMethod(Blubb,"constructor",
  [IsString,IsInt], function(s,i)
    local r;
    r := rec(wo:=s,p:=i);
    return Objectify(BlubbDenseType,r);
  end);
\end{verbatim}
}
\end{frame}

\begin{frame}[fragile]
\frametitle{The complete example, continued}
{\small
\begin{verbatim}
InstallMethod(IsShort,"for dense Blubbs",
  [IsBlubbDenseRep],
  function(bl) 
    return Length(bl!.wo) <= 5;
  end);

InstallMethod(NrLetters,"for dense Blubbs",
  [IsBlubbDenseRep],
  function(bl)
    return Length(Set(bl!.wo));
  end);

InstallMethod(ViewObj,"for dense Blubbs",
  [IsBlubbDenseRep],
  function(bl)
    Print("<a dense blubb wo=",bl!.wo,
          " p=",bl!.p,">");
  end);
\end{verbatim}
}
\end{frame}

\end{document}

