%\documentclass[handout]{beamer}
\documentclass{beamer}

\mode<presentation>
{
  %\usetheme{Warsaw}
  %\usetheme{Antibes}
  %\usetheme{Berkeley}
  %\usetheme{Copenhagen}
  %\usetheme{Hannover}
  %\usetheme{JuanLesPins}
  %\usetheme[left]{Marburg}
  %\usetheme{PaloAlto}
  %\usetheme{sidebar}
  \usetheme{CambridgeUS}
  %\useinnertheme[shadow=true]{rounded}
  %\usetheme{Singapore}
  %\usecolortheme{crane}
  %\usecolortheme{lily}
  %\usecolortheme{beetle}
  \usecolortheme{orchid}
  % oder ...
  %\setbeamercovered{transparent}
  % oder auch nicht
  %\setbeamercovered{transparent}
  % or whatever (possibly just delete it)
  \setbeamertemplate{navigation symbols}{}
  %\setbeamertemplate{blocks}[rounded][shadow=true]
  %\setbeamercolor{block body}{bg=bg=normal text.bg!70!black}
  %\setbeamercolor{subsection in sidebar}{fg=white}
}


\usepackage[british]{babel}
% oder was auch immer
\usepackage{amsmath}
\usepackage{amssymb}
\usepackage[noend]{algorithmic}
%\usepackage{algorithm}
\usepackage{stmaryrd}

\theoremstyle{definition}
%\newtheorem{lemma}[theorem]{Lemma}
\newtheorem{defprop}[theorem]{Definition/Proposition}
\newtheorem{question}[theorem]{Question}
\newtheorem{assumption}[theorem]{Assumption}
\newtheorem{construction}[theorem]{Construction}
\newtheorem{principle}[theorem]{Principle}
\newtheorem{idea}[theorem]{Idea}
\newtheorem{proposition}[theorem]{Proposition}
%\newtheorem{problem}[theorem]{Problem}
\usepackage[latin1]{inputenc}
% oder was auch immer

\usepackage{times}
\usepackage[T1]{fontenc}
% Oder was auch immer. Zu beachten ist, das Font und Encoding passen
% m�ssen. Falls T1 nicht funktioniert, kann man versuchen, die Zeile
% mit fontenc zu l�schen.

%\usepackage[mtbold,subscriptcorrection,mtpluscal]{mathtime}

%\usepackage{graphicx}
%\usepackage{rotating}

\usepackage[all]{xy}

\newcommand{\Struc}[1]{{\color{structure}#1}}
\newcommand{\Alert}[1]{{\color{alert}#1}}

\newcommand{\F}{\mathbb{F}}
\newcommand{\N}{\mathbb{N}}
\newcommand{\Z}{\mathbb{Z}}
\newcommand{\R}{\mathbb{R}}
\newcommand{\Oh}{\mathcal{O}}
\newcommand{\Aut}{\mathsf{Aut}}
\newcommand{\Stab}{\mathsf{Stab}}
\newcommand{\ob}{\mathsf{Ob}}
\newcommand{\mor}{\mathsf{Mor}}
\newcommand{\PSL}{\mathsf{PSL}}
\newcommand{\cR}{\mathcal{R}}
\newcommand\cyclic\circlearrowleft

\newcommand{\mybar}[1]{\overline{\raisebox{1.2ex}{}#1}}
\newcommand{\mybaremp}{\mybar{\ \ }}

\usepackage{calc}

\newsavebox{\linksrausbox}
\newlength{\linksrauslen}
\newcommand{\linksraus}[1]{\sbox{\linksrausbox}{#1}%
\settowidth{\linksrauslen}{\usebox{\linksrausbox}\ }%
\usebox{\linksrausbox} \begin{minipage}[t]{\textwidth-\linksrauslen}}
\newcommand{\linksrausend}{\end{minipage}\par}%\smallskip}

\usepackage{pgf,pgfarrows,pgfnodes}

%\pgfdeclareimage[width=1mm]{checkmark}{checkmark}
%\newcommand{\eofr}[1]{\vfill\vspace*{-#1mm}\hfill\pgfuseimage{checkmark}
%\par\vspace*{#1mm}\vspace*{-4mm}}

\newcommand{\GAP}{\textsf{GAP}}

\pgfdeclareimage[width=0.5in]{univstandlogo}{univstandlogo}
%\title[Computing the 2-modular characters of Fi$_{23}$] 
%          % (optional, nur bei langen Titeln n�tig)
%          {Computing the 2-modular characters of Fi$_{23}$}
\title[Finitely presented groups 2]
{Finitely presented groups 2}

\author% (optional, nur bei vielen Autoren)
{Max Neunh�ffer}

\institute[University of St Andrews] % (optional, aber oft n�tig)
{ 
\pgfuseimage{univstandlogo} \\[5mm]
%University of St Andrews  \\[2mm]
}

\date[30 July 2013] % (optional, sollte der abgek�rzte Konferenzname sein)
{LMS Short Course on Computational Group Theory \\ 29 July -- 2 August 2013}

\begin{document}
%\newcmykcolor{MyRedViolet}{0.07 0.90 0 0.34}
\begin{frame}
  \titlepage
\end{frame}

\section{Finite index subgroups}

\subsection{Theory}

\begin{frame}
Let $G := \left< X \mid R \right>$. There is a \Struc{bijection}:

\[ \begin{array}{ccc}
   \{ H \le G \mid [G:H] < \infty \}
   &\stackrel{\cong}{\longrightarrow} &
   \{ \varphi : G \to S_n \mid n \in \N, \varphi 
   \mbox{ a \Alert{grp.~hom.}} \} \\[2mm]
\pause
   H
   & \longmapsto &
   \mbox{\Alert{action} on } \{ Hg \mid g \in G \} =: H \backslash G \\[2mm]
\pause
   \varphi^{-1}(\Stab_{S_n}(1))
   & \longmapsfrom &
   \varphi \\
   \end{array} \]
\pause
which \Struc{respects}
\[ \begin{array}{ccc}
   H = K^x \mbox{ for some } x \in G
   & \Longleftrightarrow &
   \mbox{\Alert{actions} on } \{ Hg \mid g \in G\} \mbox{ and } \\
   && \{ Kg \mid g \in G \} \mbox{ \Alert{equivalent}} \\
   \end{array} \]
\pause
\centerline{The \Struc{conjugacy classes of finite index subgroups}}

\vspace*{2mm}
\centerline{are in bijection with}

\vspace*{2mm}
\centerline{the \Struc{equivalence classes of 
actions on finitely many points}.}
\end{frame}

\subsection{Coset tables}

\begin{frame}
$G$ is \Struc{finitely generated}, we describe \Struc{finite
actions} by \Alert{coset tables}:

\begin{example}[A coset table]
\vspace*{-4mm}
\begin{eqnarray*}
\mbox{Let } G &:=& \left< c,d \mid c^2 = 1 = d^3 = (cd)^8 = [c,d]^4 \right> \\
\pause
       &\cong& L_2(7):2 \cong \left< (2,4)(3,5)(6,8),
(1,2,3)(5,6,7)\right> \le S_8.
\end{eqnarray*}

\pause
\vspace*{-5mm}
\[ \begin{array}{|c||c|c|c|c|}
\hline
\mbox{Coset \#}  & c & c^{-1} & d & d^{-1} \\
\hline
\hline
1 & 1&  1&  2&  3 \\
\hline
2 & 4&  4&  3&  1 \\
\hline
3 & 5&  5&  1&  2 \\
\hline
4 & 2&  2&  4&  4 \\
\hline
5 & 3&  3&  6&  7 \\
\hline
6 & 8&  8&  7&  5 \\
\hline
7 & 7&  7&  5&  6 \\
\hline
8 & 6&  6&  8&  8 \\
\hline
\end{array} \]
\pause
Here, $H = \left< c, dcdc^{-1}d^{-1} \right>$.
\end{example}

\end{frame}

\section{Todd-Coxeter coset enumeration}

\begin{frame}
\centerline{\Huge Todd-Coxeter}
\end{frame}

\subsection{The idea}

\begin{frame}
Let $G := \left< X \mid R \right>$ and $H = \left< h_1, \ldots, h_k\right>
< G$.

\pause
\bigskip
\begin{block}{Idea of coset enumeration}
We construct the \Struc{permutation action} of $G$ on the right 
cosets of $H$.

\pause
We \Struc{give names to right cosets} and make sure that
\begin{itemize}
\item multiplication by elements of $R$ fixes all cosets, and
\pause
\item multiplication of $H$ by elements of $H$ fixes this coset.
\end{itemize}
\end{block}

\pause
A ``\Alert{name}'' of a coset is a \Struc{number and a word representing the
coset}.

\medskip
\pause
We \Struc{make up new names} and \Alert{draw conclusions} as we go and 
hope$\ldots$
\end{frame}


\subsection{A worked example}

\begin{frame}
Let $G := \left< a,b \mid a^2, b^3, abab\right>$ and $H := \left< ab
\right>$.

\vspace*{-1.5\baselineskip}
%%%%%%%%%%%%%%%%%%%%%%%%%%%%%%%%%%%%%%%%%%%%%%%%%%%%%%%%%%%%%%%%%%%
\only<1>{%
\raisebox{-1.5\baselineskip}{\begin{tabular}[t]{|l|l|c|c|c|c|}
\hline
 \#  & coset    & $a$ & $a^{-1}$ & $b$ & $b^{-1}$ \\
\hline
\hline
  1  & H        &     &          &     &          \\
     &          &     &          &     &          \\
     &          &     &          &     &          \\
     &          &     &          &     &          \\
     &          &     &          &     &          \\
\hline
\end{tabular}}
\hfill
\begin{minipage}[t]{1.5in}
\Struc{Events:} \\[1mm] \small
\mbox{}
\end{minipage}%

\medskip\rule{0mm}{2ex}%
We start with an empty table like this.
}%
%%%%%%%%%%%%%%%%%%%%%%%%%%%%%%%%%%%%%%%%%%%%%%%%%%%%%%%%%%%%%%%%%%%
\only<2>{%
\raisebox{-1.5\baselineskip}{\begin{tabular}[t]{|l|l|c|c|c|c|}
\hline
 \#  & coset    & $a$ & $a^{-1}$ & $b$ & $b^{-1}$ \\
\hline
\hline
  1  & H        &  2  &          &     &          \\
  2  & Ha       &     &     1    &     &          \\
     &          &     &          &     &          \\
     &          &     &          &     &          \\
     &          &     &          &     &          \\
\hline
\end{tabular}}
\hfill
\begin{minipage}[t]{1.5in}
\Struc{Events:} \\[1mm] \small
Def.~$2 := Ha$
\end{minipage}%

\medskip\rule{0mm}{2ex}%
We call the coset $Ha$ number $2$, a \Struc{definition}.
}%
%%%%%%%%%%%%%%%%%%%%%%%%%%%%%%%%%%%%%%%%%%%%%%%%%%%%%%%%%%%%%%%%%%%
\only<3>{%
\raisebox{-1.5\baselineskip}{\begin{tabular}[t]{|l|l|c|c|c|c|}
\hline
 \#  & coset    & $a$ & $a^{-1}$ & $b$ & $b^{-1}$ \\
\hline
\hline
  1  & H        &  2  &     2    &     &          \\
  2  & Ha       &  1  &     1    &     &          \\
     &          &     &          &     &          \\
     &          &     &          &     &          \\
     &          &     &          &     &          \\
\hline
\end{tabular}}
\hfill
\begin{minipage}[t]{1.5in}
\Struc{Events:} \\[1mm] \small
Def.~$2 := Ha$ \\
Ded.~$Haa=H$
\end{minipage}%

\medskip\rule{0mm}{2ex}%
Note $Haa = H$, since $a^2=1$, this is a \Struc{deduction}.
}%
%%%%%%%%%%%%%%%%%%%%%%%%%%%%%%%%%%%%%%%%%%%%%%%%%%%%%%%%%%%%%%%%%%%
\only<4>{%
\raisebox{-1.5\baselineskip}{\begin{tabular}[t]{|l|l|c|c|c|c|}
\hline
 \#  & coset    & $a$ & $a^{-1}$ & $b$ & $b^{-1}$ \\
\hline
\hline
  1  & H        &  2  &     2    &     &     2    \\
  2  & Ha       &  1  &     1    &  1  &          \\
     &          &     &          &     &          \\
     &          &     &          &     &          \\
     &          &     &          &     &          \\
\hline
\end{tabular}}
\hfill
\begin{minipage}[t]{1.5in}
\Struc{Events:} \\[1mm] \small
Def.~$2 := Ha$ \\
Ded.~$Haa=H$ \\
Ded.~$Hab=H$
\end{minipage}%

\medskip\rule{0mm}{2ex}%
Note $Hab = H$, since $ab\in H$, this is a \Struc{deduction}.
}%
%%%%%%%%%%%%%%%%%%%%%%%%%%%%%%%%%%%%%%%%%%%%%%%%%%%%%%%%%%%%%%%%%%%
\only<5>{%
\raisebox{-1.5\baselineskip}{\begin{tabular}[t]{|l|l|c|c|c|c|}
\hline
 \#  & coset    & $a$ & $a^{-1}$ & $b$ & $b^{-1}$ \\
\hline
\hline
  1  & H        &  2  &     2    &  3  &     2    \\
  2  & Ha       &  1  &     1    &  1  &          \\
  3  & Hb       &  4  &          &     &     1    \\
  4  & Hba      &     &     3    &     &          \\
     &          &     &          &     &          \\
\hline
\end{tabular}}
\hfill
\begin{minipage}[t]{1.5in}
\Struc{Events:} \\[1mm] \small
Def.~$2 := Ha$ \\
Ded.~$Haa=H$ \\
Ded.~$Hab=H$ \\
Def.~$3 := Hb$ \\
Def.~$4 := Hba$
\end{minipage}%

\medskip\rule{0mm}{2ex}%
Next we \Struc{define} $3 := Hb$ and $4 := Hba$.
}%
%%%%%%%%%%%%%%%%%%%%%%%%%%%%%%%%%%%%%%%%%%%%%%%%%%%%%%%%%%%%%%%%%%%
\only<6>{%
\raisebox{-1.5\baselineskip}{\begin{tabular}[t]{|l|l|c|c|c|c|}
\hline
 \#  & coset    & $a$ & $a^{-1}$ & $b$ & $b^{-1}$ \\
\hline
\hline
  1  & H        &  2  &     2    &  3  &     2    \\
  2  & Ha       &  1  &     1    &  1  &          \\
  3  & Hb       &  4  &     4    &     &     1    \\
  4  & Hba      &  3  &     3    &     &          \\
     &          &     &          &     &          \\
\hline
\end{tabular}}
\hfill
\begin{minipage}[t]{1.5in}
\Struc{Events:} \\[1mm] \small
Def.~$2 := Ha$ \\
Ded.~$Haa=H$ \\
Ded.~$Hab=H$ \\
Def.~$3 := Hb$ \\
Def.~$4 := Hba$ \\
Ded.~$Hbaa=Hb$
\end{minipage}%

\medskip\rule{0mm}{2ex}%
\Struc{Deduce} $Hbaa=Hb$.
}%
%%%%%%%%%%%%%%%%%%%%%%%%%%%%%%%%%%%%%%%%%%%%%%%%%%%%%%%%%%%%%%%%%%%
\only<7>{%
\raisebox{-1.5\baselineskip}{\begin{tabular}[t]{|l|l|c|c|c|c|}
\hline
 \#  & coset    & $a$ & $a^{-1}$ & $b$ & $b^{-1}$ \\
\hline
\hline
  1  & H        &  2  &     2    &  3  &     2    \\
  2  & Ha       &  1  &     1    &  1  &          \\
  3  & Hb       &  4  &     4    &  5  &     1    \\
  4  & Hba      &  3  &     3    &     &          \\
  5  & Hbb      &     &          &     &     3    \\
\hline
\end{tabular}}
\hfill
\begin{minipage}[t]{1.5in}
\Struc{Events:} \\[1mm] \small
Def.~$2 := Ha$ \\
Ded.~$Haa=H$ \\
Ded.~$Hab=H$ \\
Def.~$3 := Hb$ \\
Def.~$4 := Hba$ \\
Ded.~$Hbaa=Hb$ \\
Def.~$5 := Hbb$
\end{minipage}%

\medskip\rule{0mm}{2ex}%
\Struc{Define} $5 := Hbb$.
}%
%%%%%%%%%%%%%%%%%%%%%%%%%%%%%%%%%%%%%%%%%%%%%%%%%%%%%%%%%%%%%%%%%%%
\only<8>{%
\raisebox{-1.5\baselineskip}{\begin{tabular}[t]{|l|l|c|c|c|c|}
\hline
 \#  & coset    & $a$ & $a^{-1}$ & $b$ & $b^{-1}$ \\
\hline
\hline
  1  & H        &  2  &     2    &  3  &  \Alert{2} \\
  2  & Ha       &  1  &     1    &  1  &          \\
  3  & Hb       &  4  &     4    &  5  &     1    \\
  4  & Hba      &  3  &     3    &     &          \\
  5  & Hbb      &     &          &  1  &     3    \\
\hline
\end{tabular}}
\hfill
\begin{minipage}[t]{1.5in}
\Struc{Events:} \\[1mm] \small
$[\ldots]$ \\
Ded.~$Hab=H$ \\
Def.~$3 := Hb$ \\
Def.~$4 := Hba$ \\
Ded.~$Hbaa=Hb$ \\
Def.~$5 := Hbb$ \\
Ded.~$Hbbb=H$
\end{minipage}%

\medskip\rule{0mm}{2ex}%
\Struc{Deduce} $Hbbb=H$. \Alert{Thus $Hb^{-1}$ is both $2$ and $5$!}
}%
%%%%%%%%%%%%%%%%%%%%%%%%%%%%%%%%%%%%%%%%%%%%%%%%%%%%%%%%%%%%%%%%%%%
\only<9>{%
\raisebox{-1.5\baselineskip}{\begin{tabular}[t]{|l|l|c|c|c|c|}
\hline
 \#  & coset    & $a$ & $a^{-1}$ & $b$ & $b^{-1}$ \\
\hline
\hline
  1  & H        &  2  &     2    &  3  &     2    \\
  2  & Ha       &  1  &     1    &  1  &     3    \\
  3  & Hb       &  4  &     4    &  2  &     1    \\
  4  & Hba      &  3  &     3    &     &          \\
  5  & Hbb      &  -- &    --    & --  &    --    \\
\hline
\end{tabular}}
\hfill
\begin{minipage}[t]{1.5in}
\Struc{Events:} \\[1mm] \small
$[\ldots]$ \\
Def.~$3 := Hb$ \\
Def.~$4 := Hba$ \\
Ded.~$Hbaa=Hb$ \\
Def.~$5 := Hbb$ \\
Ded.~$Hbbb=H$ \\
Coi.~$5=2$
\end{minipage}%

\medskip\rule{0mm}{2ex}%
\Struc{Conclude} $Ha=Hbb$, replace $5$ by $2$: a \Struc{coincidence}.
}%
%%%%%%%%%%%%%%%%%%%%%%%%%%%%%%%%%%%%%%%%%%%%%%%%%%%%%%%%%%%%%%%%%%%
\only<10>{%
\raisebox{-1.5\baselineskip}{\begin{tabular}[t]{|l|l|c|c|c|c|}
\hline
 \#  & coset    & $a$ & $a^{-1}$ & $b$ & $b^{-1}$ \\
\hline
\hline
  1  & H        &  2  &     2    &  3  &     2    \\
  2  & Ha       &  1  &     1    &  1  &     3    \\
  3  & Hb       &  4  &     4    &  2  &     1    \\
  4  & Hba      &  3  &     3    &     &          \\
  5  & Hbb      &  -- &    --    & --  &    --    \\
\hline
\end{tabular}}
\hfill
\begin{minipage}[t]{1.5in}
\Struc{Events:} \\[1mm] \small
$[\ldots]$ \\
Def.~$3 := Hb$ \\
Def.~$4 := Hba$ \\
Ded.~$Hbaa=Hb$ \\
Def.~$5 := Hbb$ \\
Ded.~$Hbbb=H$ \\
Coi.~$5=2$
\end{minipage}%

\medskip\rule{0mm}{2ex}%
Note $Habab=H$, a \Struc{deduction} that is already known.
}%
%%%%%%%%%%%%%%%%%%%%%%%%%%%%%%%%%%%%%%%%%%%%%%%%%%%%%%%%%%%%%%%%%%%
\only<11>{%
\raisebox{-1.5\baselineskip}{\begin{tabular}[t]{|l|l|c|c|c|c|}
\hline
 \#  & coset    & $a$ & $a^{-1}$ & $b$ & $b^{-1}$ \\
\hline
\hline
  1  & H        &  2  &     2    &  3  &     2    \\
  2  & Ha       &  1  &     1    &  1  & \Alert{3}\\
  3  & Hb       &  4  &     4    &  2  &     1    \\
  4  & Hba      &  3  &     3    &  2  &          \\
  5  & Hbb      &  -- &    --    & --  &    --    \\
\hline
\end{tabular}}
\hfill
\begin{minipage}[t]{1.5in}
\Struc{Events:} \\[1mm] \small
$[\ldots]$ \\
Def.~$4 := Hba$ \\
Ded.~$Hbaa=Hb$ \\
Def.~$5 := Hbb$ \\
Ded.~$Hbbb=H$ \\
Coi.~$5=2$ \\
Ded.~$Hbab=Ha$
\end{minipage}%

\medskip\rule{0mm}{2ex}%
Use $Ha \cdot abab=Ha$, \Struc{deduce} $Hbab = Ha$.
}%
%%%%%%%%%%%%%%%%%%%%%%%%%%%%%%%%%%%%%%%%%%%%%%%%%%%%%%%%%%%%%%%%%%%
\only<12->{%
\raisebox{-1.5\baselineskip}{\begin{tabular}[t]{|l|l|c|c|c|c|}
\hline
 \#  & coset    & $a$ & $a^{-1}$ & $b$ & $b^{-1}$ \\
\hline
\hline
  1  & H        &  2  &     2    &  3  &     2    \\
  2  & Ha       &  1  &     1    &  1  &     3    \\
  3  & Hb       &  3  &     3    &  2  &     1    \\
  4  & Hba      &  -- &    --    & --  &    --    \\
  5  & Hbb      &  -- &    --    & --  &    --    \\
\hline
\end{tabular}}
\hfill
\begin{minipage}[t]{1.5in}
\Struc{Events:} \\[1mm] \small
$[\ldots]$ \\
Ded.~$Hbaa=Hb$ \\
Def.~$5 := Hbb$ \\
Ded.~$Hbbb=H$ \\
Coi.~$5=2$ \\
Ded.~$Hbab=Ha$ \\
Table closed.
\end{minipage}%

\medskip\rule{0mm}{2ex}%
Conclude $Hba=Hb$, replace $4$ by $3$. \Struc{Table is closed}.
}%
%%%%%%%%%%%%%%%%%%%%%%%%%%%%%%%%%%%%%%%%%%%%%%%%%%%%%%%%%%%%%%%%%%%

\bigskip
\uncover<13->{Indeed, we have found a \Struc{permutation representation}
on $3$~points. The subgroup $H$ fixes the first point.}

\medskip
\uncover<14->{Since we have
\Struc{checked all relations} we have found a \Alert{group homomorphism} 
from $G$ to $S_3$.}
\end{frame}

\subsection{GAP session}

\begin{frame}

\centerline{\url{http://tinyurl.com/MNGAPsess/GAP\_FP\_4.g}}
% GAP session showing some coset enumerations

\end{frame}

\subsection{Strategies}

\begin{frame}
\begin{itemize}
\item 
We had \Struc{a lot of choice} as to what we \Alert{define next}.

\pause
\item 
We \Struc{sort out coincidences} before we \Alert{define new cosets}.

\pause
\item
\Struc{Direct deductions} are preferable to \Alert{random definitions}.

\pause
\item A \Struc{definition algorithm} is called a \Alert{strategy}.
\end{itemize}

\pause
There are many strategies, for example:
\begin{enumerate}
\item \Alert{HLT} (Haselgrove-Leech-Trotter): define cosets to \Struc{scan
relators}
\pause
\item \Alert{Felsch}: define cosets to \Struc{fill rows of the table}
\end{enumerate}

\pause
\bigskip
\Struc{However:}

\bigskip
\centerline{\textbf{No strategy is always optimal.}}

\bigskip
\pause
\centerline{\textbf{Runtime and memory usage vary enormously with the
strategy.}}
\end{frame}

\subsection{Properties of the algorithm, termination}

\begin{frame}
The Todd-Coxeter algorithm has the following features:

\begin{itemize}
\item \Alert{If it terminates}, it \Struc{proves} that $[G:H]$ is finite
and it \Struc{constructs} the permutation action of $G$ on the right cosets
of $H$.
\end{itemize}

\pause
\begin{theorem}[Termination of the Todd-Coxeter procedure]
Assume \Alert{$[G:H] < \infty$} and 
a \Alert{deterministic strategy} with:
\begin{itemize}
\item all entries will \Struc{eventually be filled},
\item all relators will \Struc{eventually be scanned} from each coset, and
\item all subgroup generators will \Struc{eventually be scanned} from
coset \#1.
\end{itemize}
Then the Todd-Coxeter procedure \Struc{terminates eventually}.
\end{theorem}

\pause
\begin{itemize}
\item \Alert{No limit} on \Struc{memory} and \Struc{runtime} is known
a priori.
\pause
\item A completed coset enumeration with $H=\{1\}$ \Struc{proves} $G$ to be 
\Struc{finite} and \Struc{determines the order}.
\pause
\item An \Struc{unfinished} coset enumeration \Alert{proves nothing
whatsoever}.
\end{itemize}
\end{frame}


\section{Enumerating low index subgroups}

\begin{frame}
\centerline{\Huge Low index}
\end{frame}

\subsection{The idea}

\begin{frame}
Let $G := \left< X \mid R \right>$.

\pause
\bigskip
\begin{block}{Idea of the low index procedure}
We want to construct a \Struc{permutation action} of $G$ on
$k$ points.
\pause
This is equivalent to the action on the cosets of a subgroup $H$ of index
$k$.

\pause
We \Struc{start with an empty coset table with $k$ rows} and
\begin{itemize}
\item \Struc{try out} all possibilities to fill it in (\Alert{finite!}),
\pause
\item \Struc{check} that all elements of $R$ act trivially,
\pause
\item \Struc{use backtrack search},
\pause
\item \Struc{determine} the point stabiliser $H$ in each case, and
\pause
\item \Struc{remove} equivalent actions (conjugate subgroups $H$).
\end{itemize}
\end{block}

\pause
\medskip
$\Longrightarrow$ This \Alert{very quickly} becomes \Alert{impractical} 
for larger $k$.
\end{frame}


\subsection{A worked example}

\begin{frame}
\frametitle{Low index: an example}

Let $G := \left< a,b \mid a^2, b^3, abab\right>$ and $k=3$.

%\vspace*{-1.5\baselineskip}
%%%%%%%%%%%%%%%%%%%%%%%%%%%%%%%%%%%%%%%%%%%%%%%%%%%%%%%%%%%%%%%%%%%
\only<1>{%
%\raisebox{-1.5\baselineskip}{
\begin{tabular}[t]{|l|l|c|c|c|c|}
\hline
 \#  &  $a$ & $b$ & $b^{-1}$ \\
\hline
\hline
  1  &      &     &          \\
  2  &      &     &          \\
  3  &      &     &          \\
\hline
\end{tabular}
\hfill
\begin{minipage}[t]{2.5in}
\Struc{Guesses:} \\[1mm] \small
\mbox{}
\end{minipage}%

\medskip\rule{0mm}{2ex}%
We start with an empty table like this.
}%
%%%%%%%%%%%%%%%%%%%%%%%%%%%%%%%%%%%%%%%%%%%%%%%%%%%%%%%%%%%%%%%%%%%
\only<2>{%
%\raisebox{-1.5\baselineskip}{
\begin{tabular}[t]{|l|l|c|c|c|c|}
\hline
 \#  &  $a$ & $b$ & $b^{-1}$ \\
\hline
\hline
  1  &   1  &     &          \\
  2  &      &     &          \\
  3  &      &     &          \\
\hline
\end{tabular}
\hfill
\begin{minipage}[t]{2.5in}
\Struc{Guesses:} \\[1mm] \small
$1a = 1$
\end{minipage}%

\medskip\rule{0mm}{2ex}%
We first assume $1a = 1$. Nothing follows.
}%
%%%%%%%%%%%%%%%%%%%%%%%%%%%%%%%%%%%%%%%%%%%%%%%%%%%%%%%%%%%%%%%%%%%
\only<3>{%
%\raisebox{-1.5\baselineskip}{
\begin{tabular}[t]{|l|l|c|c|c|c|}
\hline
 \#  &  $a$ & $b$ & $b^{-1}$ \\
\hline
\hline
  1  &   1  &  2  &          \\
  2  &      &     &    1     \\
  3  &      &     &          \\
\hline
\end{tabular}
\hfill
\begin{minipage}[t]{2.5in}
\Struc{Guesses:} \\[1mm] \small
$1a = 1$ \\
$1b = 2$ (wlog)
\end{minipage}%

\medskip\rule{0mm}{2ex}%
$1b=1$ would be intransitive, so (wlog) $1b = 2$.
}%
%%%%%%%%%%%%%%%%%%%%%%%%%%%%%%%%%%%%%%%%%%%%%%%%%%%%%%%%%%%%%%%%%%%
\only<4>{%
%\raisebox{-1.5\baselineskip}{
\begin{tabular}[t]{|l|l|c|c|c|c|}
\hline
 \#  &  $a$ & $b$ & $b^{-1}$ \\
\hline
\hline
  1  &   1  &  2  &  \Alert{2}        \\
  2  &      &\Alert{1}     &    1     \\
  3  &      &     &          \\
\hline
\end{tabular}
\hfill
\begin{minipage}[t]{2.5in}
\Struc{Guesses:} \\[1mm] \small
$1a = 1$ \\
$1b = 2$ (wlog) \\
\Alert{$2b=1$}
\end{minipage}%

\medskip\rule{0mm}{2ex}%
From $2b=1$ would follow $1bbb=2$, a contradiction.
}%
%%%%%%%%%%%%%%%%%%%%%%%%%%%%%%%%%%%%%%%%%%%%%%%%%%%%%%%%%%%%%%%%%%%
\only<5>{%
%\raisebox{-1.5\baselineskip}{
\begin{tabular}[t]{|l|l|c|c|c|c|}
\hline
 \#  &  $a$ & $b$ & $b^{-1}$ \\
\hline
\hline
  1  &   1  &  2  &          \\
  2  &      &  3  &    1     \\
  3  &      &     &    2     \\
\hline
\end{tabular}
\hfill
\begin{minipage}[t]{2.5in}
\Struc{Guesses:} \\[1mm] \small
$1a = 1$ \\
$1b = 2$ (wlog)
\end{minipage}%

\medskip\rule{0mm}{2ex}%
So we backtrack and conclude $2b=3$.
}%
%%%%%%%%%%%%%%%%%%%%%%%%%%%%%%%%%%%%%%%%%%%%%%%%%%%%%%%%%%%%%%%%%%%
\only<6>{%
%\raisebox{-1.5\baselineskip}{
\begin{tabular}[t]{|l|l|c|c|c|c|}
\hline
 \#  &  $a$ & $b$ & $b^{-1}$ \\
\hline
\hline
  1  &   1  &  2  &    3     \\
  2  &      &  3  &    1     \\
  3  &      &  1  &    2     \\
\hline
\end{tabular}
\hfill
\begin{minipage}[t]{2.5in}
\Struc{Guesses:} \\[1mm] \small
$1a = 1$ \\
$1b = 2$ (wlog)
\end{minipage}%

\medskip\rule{0mm}{2ex}%
It follows that $3b=1$ for $1bbb=1$ to hold.
}%
%%%%%%%%%%%%%%%%%%%%%%%%%%%%%%%%%%%%%%%%%%%%%%%%%%%%%%%%%%%%%%%%%%%
\only<7>{%
%\raisebox{-1.5\baselineskip}{
\begin{tabular}[t]{|l|l|c|c|c|c|}
\hline
 \#  &  $a$ & $b$ & $b^{-1}$ \\
\hline
\hline
  1  &   1  &  2  &    3     \\
  2  & \Alert{2} &  3  &    1     \\
  3  & \Alert{3} &  1  &    2     \\
\hline
\end{tabular}
\hfill
\begin{minipage}[t]{2.5in}
\Struc{Guesses:} \\[1mm] \small
$1a = 1$ \\
$1b = 2$ (wlog)
\end{minipage}%

\medskip\rule{0mm}{2ex}%
$2a=2$ would imply $3a=3$ and then $1abab=3$.
}%
%%%%%%%%%%%%%%%%%%%%%%%%%%%%%%%%%%%%%%%%%%%%%%%%%%%%%%%%%%%%%%%%%%%
\only<8->{%
%\raisebox{-1.5\baselineskip}{
\begin{tabular}[t]{|l|l|c|c|c|c|}
\hline
 \#  &  $a$ & $b$ & $b^{-1}$ \\
\hline
\hline
  1  &   1  &  2  &    3     \\
  2  &   3  &  3  &    1     \\
  3  &   2  &  1  &    2     \\
\hline
\end{tabular}
\hfill
\begin{minipage}[t]{2.5in}
\Struc{Guesses:} \\[1mm] \small
$1a = 1$ \\
$1b = 2$ (wlog)
\end{minipage}%

\medskip\rule{0mm}{2ex}%
Thus we have $2a=3$ and everything is good.
}%
%%%%%%%%%%%%%%%%%%%%%%%%%%%%%%%%%%%%%%%%%%%%%%%%%%%%%%%%%%%%%%%%%%%

\bigskip
\uncover<9->{%
For this solution $H=\left<a\right>$.}

\bigskip
\uncover<10->{%
To go on, we would change the assumption $1a=1$ to $1a=2$ (wlog)
and continue the search.}

\end{frame}

\subsection{GAP session}

\begin{frame}

\centerline{\url{http://tinyurl.com/MNGAPsess/GAP\_FP\_5.g}}
% GAP session showing low index computation, also with iterator

\end{frame}

\subsection{Removing duplicates}

\begin{frame}
\begin{definition}[Standardised coset table]
A coset table is called \Struc{standardised}, if reading it row by row and
left to right finds new numbers in order 2, 3, $\ldots$, $n$.
\end{definition}

\pause
\begin{tabular}[c]{|l|l|c|c|c|c|}
\hline
 \#  &  $a$ & $b$ & $b^{-1}$ \\
\hline
\hline
  1  &   1  &  2  &    3     \\
  2  &   3  &  3  &    1     \\
  3  &   2  &  1  &    2     \\
\hline
\end{tabular} is standardised,
\pause
\begin{tabular}[c]{|l|l|c|c|c|c|}
\hline
 \#  &  $a$ & $b$ & $b^{-1}$ \\
\hline
\hline
  1  &   1  &  3  &    2     \\
  2  &   3  &  1  &    3     \\
  3  &   2  &  2  &    1     \\
\hline
\end{tabular} is not.

\pause
\begin{proposition}
For $G := \left< X \mid R \right>$, the set of \Struc{subgroups of index
$n$} is in bijection with the set of
\Struc{standardised coset tables with $n$ rows}.
\pause
If $[G:H]=n$, the \Struc{conjugates $\{H^x\mid x\in G\}$} are the
stabilisers of the points $\{Hx \mid x \in G\}$.
\end{proposition}

\pause
Therefore, the \Struc{low index procedure}
\begin{itemize}
\item only \Struc{enumerates} \Alert{standardised coset tables},
\pause
\item and \Struc{discards} those which are not \Alert{lexicographically
least} among the renumbered ones, \pause this \Struc{gets rid of duplicates}.
\end{itemize}
\end{frame}


\end{document}

