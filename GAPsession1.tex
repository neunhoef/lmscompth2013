\documentclass[12pt]{article}

\pagestyle{empty}
\parindent0pt
%\usepackage{charter}
\usepackage{times}
\usepackage[a4paper,margin=1in]{geometry}
\newcommand{\GAP}{\textsf{GAP}}

\begin{document}
\begin{center}
\large LMS Short Course on Computational Group Theory

\Large Lab session

\large Getting to know \GAP
\end{center}

There are hints on this sheet, however, if you get \textbf{stuck}, then
\textbf{ask} someone.

\begin{enumerate}
\setlength{\parskip}{0pt}
\item Find out how to start \GAP, type in commands and compute
$123456 \cdot 654321$.

\textbf{Hints:} Type \texttt{gap} or \texttt{gap4} at the command prompt,
end your \GAP\ commands with a semicolon (\texttt{;}).

\item Find out how to use \GAP's help system, look up the \texttt{Orbit}
command.

\textbf{Hints:} Type a question mark (\texttt{?}) followed by a word
to find a manual section heading beginning with that word. Type two
question marks (\texttt{??}) followed by a word to find all manual
section headings containing that word.

\item Read about the \texttt{InputLogTo} command. Log your input to a file.
\item Find out whether the product of the two permutations $(1,2)$ and
$(2,3)$ is $(1,2,3)$ or $(1,3,2)$. This tells you whether \GAP\ prefers
to act from the left or right.

\item Compute the factorial of $1000$, that is $1000! = 1 \cdot 2 \cdots
999 \cdot 1000$.

\textbf{Hints:} You can use the library function. Later we will write
a function to do this on our own.
\item Try to divide by zero and see what happens.

\textbf{Hints:} Read about the so-called ``break loop'' under
\texttt{?break loop}. 
\item Find out how to edit a text file on your computer, call it
``\texttt{test.g}'' and read it into \GAP\ using the \texttt{Read} command.
\item It is now time to write your first function, try this one:
{\small \begin{verbatim}
MyFirst := function(a,b)
  local c;
  c := a^2 + b^2;
  Print("It worked: ",a," and ",b," gives ",c,"\n");
  return c;
end;
\end{verbatim}}
\item Write a function computing factorials.

\textbf{Hints:} Read about the \texttt{for} and \texttt{while} and
\texttt{if} commands.
\item Put an \texttt{Error} statement into the above function.

\textbf{Hints:} \GAP\ enters a break loop during execution.
This is very useful for debugging.
\item Compute the orbit of the pair $[3,6]$ under the action on pairs
of the permutation group generated by the three permutations

\vspace*{-5mm}
\begin{center}{\small
$(1,2,3,4,5,6,7,8,9,10,11)$, $(3,7,11,8)(4,10,5,6)$ and
$(1,12)(2,11)(3,6)(4,8)(5,9)(7,10)$.}
\end{center}

\vspace*{-5mm}
\item Write a program to find a solution to the ``$8$-queens problem'': How
can you place $8$ queens on a chess board such that none of them attacks
another one (i.e.~no two are in the same row, column or diagonal). 

\textbf{Hints:} You probably have to learn about \textbf{lists} for this.
Try \texttt{?lists} for this.
\end{enumerate}
\end{document}
