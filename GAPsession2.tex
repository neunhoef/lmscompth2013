\documentclass[12pt]{article}

\pagestyle{empty}
\parindent0pt
%\usepackage{charter}
\usepackage{times}
\usepackage[a4paper,margin=1in]{geometry}
\newcommand{\GAP}{\textsf{GAP}}
\newcommand{\Stab}{\mbox{Stab}}

\begin{document}
\begin{center}
\large LMS Short Course on Computational Group Theory

\Large Lab session

\large Getting to know \GAP --- first steps with permutation groups
\end{center}

{\small
There are hints on
this sheet. We suggest that you first try to solve the exercises without
using the hints, however, if you get \textbf{stuck} with one, then first
read only the first hint and try again. If this does not help, try the
second hint and so on. Finally, if nothing helps, \textbf{ask} someone.}

\begin{enumerate}
\setlength{\parskip}{0pt}
\item Let $G$ be the group generated by the following two permutations:

%\vspace*{-5mm}
\begin{center}{\small
$(1,10)(2,3,6,9,5,8,4,11)$ and $(1,2,5,9)(7,10,11,8)$.}
\end{center}

%\vspace*{-5mm}
We first want to analyse the action of this group on $M := \{1,2,\ldots,11\}$:

Find, using \GAP, the largest $k$ such that $G$ acts $k$-transitively
on $M$.

\smallskip
\textbf{Hint 0}: To enter the group, use the \texttt{Group} command ($\to$
\texttt{?Group}).

\smallskip
\textbf{Hint 1}: Compute the orbit of $1$ under $G$ and decide, whether or
not the action is transitive ($\to$ \texttt{?Orbit} and
$\to$ \texttt{?IsTransitive}).

\smallskip
\textbf{Hint 2}: Compute the stabiliser of $1$ in $G$ and apply the same
method to it, of course using a different starting point
($\to$ \texttt{?Stabilizer})

\smallskip
\textbf{Hint 3}: Repeat.
%%%%%%%%%%%%%%%%%%%%%%%%%%%%%%%%%%%%%%%%%%%%%%%%%%%%%%%%%%%%%
\item For the $k$ you found in 1: does $G$ act \textbf{sharply}
$k$-transitively?

\smallskip
\textbf{Hint 1:} Look at the last stabiliser you computed.
%%%%%%%%%%%%%%%%%%%%%%%%%%%%%%%%%%%%%%%%%%%%%%%%%%%%%%%%%%%%%
\item Let's analyse the structure of this group a bit:
Compute the group order ($\to$ \texttt{?Size}).
%%%%%%%%%%%%%%%%%%%%%%%%%%%%%%%%%%%%%%%%%%%%%%%%%%%%%%%%%%%%%
\item Compute the center of this group.

\smallskip
\textbf{Hint 1}: $\to$ \texttt{?Center}
%%%%%%%%%%%%%%%%%%%%%%%%%%%%%%%%%%%%%%%%%%%%%%%%%%%%%%%%%%%%%
\item Compute the derived subgroup of this group.

\smallskip
\textbf{Hint 1}: $\to$ \texttt{?DerivedSubgroup}
%%%%%%%%%%%%%%%%%%%%%%%%%%%%%%%%%%%%%%%%%%%%%%%%%%%%%%%%%%%%%
\item Check if this group is simple.

\smallskip
\textbf{Hint 1}: $\to$ \texttt{?IsSimple}
%%%%%%%%%%%%%%%%%%%%%%%%%%%%%%%%%%%%%%%%%%%%%%%%%%%%%%%%%%%%%
\item Compute the $2$-, $3$-, $5$- and $11$-Sylow subgroups of $G$.

\smallskip
\textbf{Hint 1}: $\to$ \texttt{?SylowSubgroup}
%%%%%%%%%%%%%%%%%%%%%%%%%%%%%%%%%%%%%%%%%%%%%%%%%%%%%%%%%%%%%
\item Compute the stabiliser of $1$ in $G$ and apply the above methods to
it to find out something about its structure.

\smallskip
\textbf{Hint 1}: $\to$ \texttt{?Stabiliser}
%%%%%%%%%%%%%%%%%%%%%%%%%%%%%%%%%%%%%%%%%%%%%%%%%%%%%%%%%%%%%
\item Let's study the derived subgroup $D$ of $\Stab_G(1)$: Confirm
that it is a simple group of order $360$.
%%%%%%%%%%%%%%%%%%%%%%%%%%%%%%%%%%%%%%%%%%%%%%%%%%%%%%%%%%%%%
\item We suspect that this might be isomorphic to the alternating group
$A_6$ on $6$ points. Verify this with \GAP\ and compute an explicit
isomorphism. 

Compute the images of the generators of $D$ under this
isomorphism and the preimages of the standard generators of $A_6$.

\smallskip
\textbf{Hint 1}: $\to$ \texttt{?IsomorphismGroups}

\smallskip
\textbf{Hint 2}: This gives you a \GAP\ object representing an isomorphism.
You can access the generators of a group with \texttt{GeneratorsOfGroup}.
You can map elements using \texttt{ImageElm} and compute preimages
with \texttt{PreImage}.

%%%%%%%%%%%%%%%%%%%%%%%%%%%%%%%%%%%%%%%%%%%%%%%%%%%%%%%%%%%%%
\item Find out what the following command does and why it does this
(assuming that the above group $D$ is stored in the variable \texttt{D}):

\centerline{\texttt{List(GeneratorsOfGroup(D),x->ImageElm(iso,x));}}

\smallskip
\textbf{Hint 1}: $\to$ \texttt{?List}

\smallskip
\textbf{Hint 2}: $\to$ \texttt{?arrow notation}
%%%%%%%%%%%%%%%%%%%%%%%%%%%%%%%%%%%%%%%%%%%%%%%%%%%%%%%%%%%%%
\item We want to construct this isomorphism in another way.
To this end, understand the following sequence of commands:

\begin{verbatim}
c := ConjugacyClassesMaximalSubgroups(D);
List(c,Size);
r := List(c,Representative);
List(r,Size);
\end{verbatim}
%%%%%%%%%%%%%%%%%%%%%%%%%%%%%%%%%%%%%%%%%%%%%%%%%%%%%%%%%%%%%
\item The above computation gave you two subgroups of index $6$.
Compute the two actions of $D$ on the right cosets of them.

\smallskip
\textbf{Hint 1}: $\to$ \texttt{FactorCosetAction}
%%%%%%%%%%%%%%%%%%%%%%%%%%%%%%%%%%%%%%%%%%%%%%%%%%%%%%%%%%%%%
\item Derive explicitly an automorphism of $D$ which does not
come from conjugation in the symmetric group $S_6$.

\end{enumerate}
\end{document}
